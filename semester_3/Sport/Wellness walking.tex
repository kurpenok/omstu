\documentclass[14pt, a4paper]{article}

\usepackage[T2A]{fontenc}
\usepackage[utf8]{inputenc}
\usepackage[english,russian]{babel}

\begin{document}
    \pagestyle{empty}

    \begin{center}
        Министерство науки и высшего образования Российской Федерации

        Федеральное государственно автономное образовательное учреждение высшего образования

        <<Омский государственный технический университет>>

        \vspace{1cm}
        Факультет информационных технологий и компьютерных систем

        Кафедра <<Прикладная математика и фундаметральная информатика>>

        \vspace{3cm}
        \textbf{Лабораторная работа}

        по дисциплине <<Физическая культура>>
    \end{center}
    
    \vspace{3cm}
    \begin{flushright}
        \begin{tabular}{ r r }
            Студента & Курпенова Куата Ибраимовича \\
            \cline{2-2}
            & \tiny{фамилия, имя, отчество полностью} \\

            Курс & 2, группа ФИТ-212 \\
            \cline{2-2}
            Направление & 02.03.02 Прикладная математика \\
            \cline{2-2}
            & и фундаментальная информатика \\
            \cline{2-2}
            & \tiny{код, наименование} \\
            
            Руководитель & доц., канд. пед. наук, доцент \\
            \cline{2-2}
            & \tiny{должность, ученая степень, звание} \\
            & Сафонова Ж. Б. \\
            \cline{2-2}
            & \tiny{фамилия, инициалы} \\
            
            Выполнил & \\
            \cline{2-2}
            & \tiny{дата, подпись студента} \\
            
        \end{tabular}
    \end{flushright}
    
    \vspace*{\fill}
    \begin{center}
        Омск 2022
    \end{center}

    \newpage

    В наше экстремальное время погони за деньгами и высокими спортивными результатами «физкультура для здоровья» ушла как для государства, так и для множества платных спортивных организаций на второй  план. Для людей, которым по состоянию здоровья показана лечебная физкультура; для лиц пожилого возраста; для тех, кто не  может по разным причинам позволить себе посещать спортивные секции и клубы, а также для большинства современных людей, которых технический  прогресс на долгое время усадил за рабочий стол, стал актуальным вопрос о том, как находиться в хорошей физической, а значит продуктивной рабочей форме с наименьшими затратами средств и времени. К числу «обреченных» на «сидячий» образ  жизни, относятся, к сожалению, и студенты. Особенно снижается их двигательная активность, как пишет преподаватель Сибирского государственного технологического университета А.П. Бызов, в период сдачи экзаменов на завершающем этапе первого полугодия.

    Наиболее простым, материально незатратным и доступным каждому средством является оздоровительная ходьба. Она относится к продолжительным умеренно интенсивным физическим упражнениям, объединенным под общим названием аэробные упражнения. К ним относятся также бег, плавание, лыжи, велосипед и ритмическая гимнастика, если она выполняется продолжительное время и непрерывно. Именно анаэробные упражнения считаются лучшими на пути укрепления здоровья, поддержания активного жизненного тонуса, повышения общей выносливости и работоспособности.

    Аэробными наука называет процессы образования энергии в организме исключительно за счет содержащегося в воздухе кислорода (греч. «аэр» – воздух) и окисления им жиров и углеводов. К разделу «аэробных» относят только те упражнения, при которых в работу вовлекается большая  часть мышц человеческого тела (примерно 2/3 всей мышечной массы), а продолжительность непрерывного выполнения этих упражнений составляет  не менее 20-30 минут. Такая непрерывная и продолжительная работа организма обеспечивается энергией в основном за счет аэробных процессов, то есть за счет «сгорания» жиров и углеводов.
    
    Среди аэробных упражнений долгое время самым популярным был бег трусцой. Затем его потеснила ритмическая гимнастика – «аэробика», которая может быть причислена к разряду аэробных только, если выполняется без перерывов на отдых по крайней мере 20-30 минут (только при этих условиях  возникает аэробная нагрузка). Если делать остановки для разучивания упражнений или отдыха, то это, всего лишь, обычная гимнастика.  Но такие занятия подходят не всем.
    
    Дело в том, что наибольшего оздоровительного эффекта можно добиться  только при условии занятий около часа в день при пульсе около 130-140  ударов в минуту. Об этом пишет кардиохирург и пропагандист здорового  образа жизни академик Николай Михайлович Амосов в книге «Раздумья  о  здоровье»: «В большинстве болезней виновата не природа, не общество, а  только сам человек. Чаще всего он болеет от лени и жадности, но иногда и от неразумности. Но природа милостива: достаточно 20-30 минут занятий физкультурой в день, но такой, чтобы задохнуться, вспотеть и чтобы пульс  участился вдвое. Если это время удвоить, то будет вообще отлично».
    
    Но не каждый сможет бежать 30-60 минут непрерывно и делать так ежедневно долгие годы. По статистике больше 50% начавших заниматься оздоровительным бегом отказались от него из-за болей в  мышцах и связках. Те же 50% отсева были  в группах ритмической гимнастики. Американские врачи, кроме того, обнаружили, что перегрузки при аэробике приводят у женщин к понижению в организме кальция и гормона эстрогена, что ведет к болезням позвоночника и бесплодию. Что касается оздоровительной ходьбы, то она обладает рядом  преимуществ:
    
    \begin{itemize}
        \item Доступность. Не нужно, как для бега, искать тропу с мягким почвенным покровом, чтобы уберечь ноги от ушибов и травм, особенно при избыточном весе. При ходьбе нет фазы полета и моментов приземления, которые являются причиной травматизма при беге. Если бег запрещен при болезнях сердца, некоторых формах диабета, болезнях почек, при которых  нельзя допускать тряски организма, то ходьбу врач может разрешить, так как здесь легко регулировать нагрузку. Так с сентября 2016 года по март 2017 года в Нижегородском государственном инженерно-экономическом университете (г. Княгинино)  проводился эксперимент по применению оздоровительной ходьбы на занятиях физкультурой в специальной медицинской группе. Среди студентов были страдающие, в том числе, хроническим пиелонефритом.  Эксперимент показал, что оздоровительная ходьба улучшила их функциональное состояние и соматическое здоровье. К такому же выводу пришли педагоги Омского юридического института, проводившие в течение нескольких лет наблюдения за студентами специальной медицинской группы, занимающимися оздоровительной ходьбой. При этом педагоги особо отметили желание студентов «повторять прогулки, превращать их в закономерность» и сделали вывод о формировании у них с помощью оздоровительной ходьбы потребности в движении на многие годы. Интерес к самостоятельной форме занятий ходьбой, стремление учащихся к самосовершенствованию и здоровому образу жизни в условиях постоянной нехватки времени отметил и А.П. Бызов, практиковавший оздоровительную ходьбу среди студентов Сибирского государственного технологического университета  г. Красноярска в период экзаменационной сессии и зимних каникул.

        \item Удобное время занятий. Ходьбой можно заниматься в режиме трудового  дня – пешком на работу, на учебу, прогулка в обеденный перерыв.
        
        \item Возможность регулировать нагрузки. Сама форма движений является ограничителем нагрузки в начале занятий ходьбой.
        
        \item Возможность преодолевать условности. Бегущий человек привлекает внимание – быстро идущий человек просто куда-то спешит.
        
        \item Возможность продолжительной нагрузки. Этим фактором  определяется эффективность тренировки сердечно-сосудистой системы. Гораздо проще для неподготовленного человека 30 минут непрерывно пройти, чем пробежать.
        
        \item Возможность совмещать умственную деятельность и ходьбу. Многие работники творческого труда  отмечали, что ходьба оживляет деятельность ума. Об этом писали Жан-Жак Руссо, Л.Н. Толстой, А.С. Пушкин, проходивший пешком расстояние от Царского Села до Петербурга (около 18  км).  Ходьба –  высокоавтоматизированный навык. На привычной трассе мозг имеет возможность отвлекаться для продуктивной работы.
        
        \item Возможность снимать нервное напряжение. Во время физических упражнений в организме синтезируются гормоны–эндофрины, благотворно влияющие на психику.
        
        \item Антисклеротический эффект. У больных коронарным атеросклерозом в крови малое количество липопротеинов высокой плотности. Делая циклические упражнения, можно избавиться от этого недуга. Поскольку атеросклерозом страдают, в основном, пожилые люди, из циклических упражнений им обычно доступна только ходьба. Научные данные свидетельствуют, что для антисклеротического эффекта достаточна нагрузка на уровне 140 ударов пульса в минуту, а после 50 лет достаточно и 130. Вышеперечисленное объясняет, почему ходьба стала сегодня для многих основой здорового образа жизни.
    \end{itemize}

    Средняя скорость обычной ходьбы составляет 5-6 километров в час. Тренированные ходоки могут развивать скорость около 8-10 км/ч.

    Как же стать хорошо тренированным ходоком? Автор брошюры «Оздоровительная ходьба» И.А. Гайс предлагает многолетнюю программу занятий оздоровительной ходьбой, разделив ее на 3 этапа: подготовительный (продолжительность 2-3 месяца) этап, этап повышения скорости ходьбы, и, наконец, этап дальнейшего совершенствования приемов ходьбы и поддержания необходимой для этого физической  формы.

    На  подготовительном этапе предлагается сначала обратиться за советом к врачу, потом определить время занятий, подобрать одежду и обувь, измерить продолжительность трассы, разучить упражнения скороходов, научиться определять скорость ходьбы и освоить методы самоконтроля. Каждый день нужно проходить пешком все большую часть пути в учебное заведение или на работу и обратно. Раз в неделю устраивать одно специальное занятие в лесу или парке. В качестве обуви рекомендуются кроссовки. Обувь должна иметь прочный задник и небольшой каблучок или утолщение вместо него (при литой подошве), желательно наличие супинатора. Не стоит использовать кеды, обувь с острыми мысками, открытой пяткой. Зимой необходимо надевать махровые шерстяные, а летом хлопчатобумажные носки, которые хорошо впитывают пот. Выбранная одежда, как спортивная, так и повседневная, должна обеспечивать свободу движений. Тренировочный  костюм лучше выбрать хлопчатобумажный или полушерстяной. Когда солнечно и жарко пригодится кепка с козырьком и 2 носовых платка. Трассу желательно выбирать на ровной незагазованной местности, определить ее  протяженность (шагами, по карте и т.д.). Зная длину трассы и время его прохождения, легко определить среднюю скорость ходьбы. Для лучшей ориентации во время ходьбы  можно использовать таблицу:

    \begin{center}
        \begin{tabular}{ | c | c | }
            \hline
            Скорость & Время, необходимое для прохождения 1 км \\
            \hline
            5 км/ч & 12 мин \\
            \hline
            6 км/ч & 10 мин \\
            \hline
            7 км/ч & 8 мин 34 сек \\
            \hline
            8 км/ч & 7 мин 30 сек \\
            \hline
            9 км/ч & 6 мин 40 сек \\
            \hline
            10 км/ч & 6 мин \\
            \hline
        \end{tabular}
    \end{center}

    Чтобы определить абсолютную скорость ходьбы нужно выбрать отрезок ровной аллеи длиной 500 метров, быстро пройти его 2-3 раза, фиксируя время, среднее время удвоить. Получится время прохождения одного километра. Разделить 1 час (3600 секунд) на среднее время. Получится абсолютная скорость.

    Начинающим ходокам необходимо освоить методы самоконтроля за степенью нагрузки. Если при ходьбе нет обильного потоотделения, одышки, самочувствие хорошее – значит нагрузка оптимальная. Способность поддерживать во время ходьбы беседу достаточно длинными фразами (т.н. разговорная скорость) тоже говорит об оптимальной нагрузке. Как установили физиологи, при ходьбе с разговорной скоростью потребление организмом кислорода составляет 60-70% от максимально возможного. Именно такой уровень нагрузки  наиболее благоприятен для тренировки сердечно-сосудистой системы.

    Другим методом контроля является подсчет ударов пульса. Это легко сделать по системе чехословацкого врача Иржи Квапилика. Из 180 необходимо вычесть свой возраст в годах. Полученное число будет количеством ударов пульса в минуту, которое нельзя превышать во время ходьбы. Ниже приводится таблица для контроля верхней и нижней границы  пульса при максимальной и минимальной нагрузке:

    \begin{center}
        \begin{tabular}{ | c | c | c | }
            \hline
            Возраст (лет) & \multicolumn{2}{ c }{Частота пульса (уд в мин)} \\
            \hline
            & min & max \\
            \hline
            20 & 134 & 160 \\
            \hline
            30 & 129 & 150 \\
            \hline
            40 & 124 & 140 \\
            \hline
            50 & 118 & 130 \\
            \hline
            60 & 113 & 120 \\
            \hline
            70 & 108 & 110 \\
            \hline
        \end{tabular}
    \end{center}

    Для точного подсчета ударов пульса в ходе занятий нужно запомнить  необходимое количество ударов за 10 секунд. Необходимо проверять скорость восстановления пульса до нормы после нагрузок. Согласно разработкам Всероссийского научно-исследовательского института физической культуры и спорта нормы времени для восстановления рекомендуются следующие: если частота ударов пульса в течение 10 минут после занятия  снижается на 30-40%, то нагрузка была умеренной; если на 20-30% – повышенной; а если только на 10-20% и менее – то большой.

    В начале подготовительного этапа рекомендуется ставить перед собой посильные задачи (особенно если присутствуют избыточный вес, сутулая осанка, дефекты походки и т.п.). К утренней зарядке прибавить упражнения скорохода и хотя бы часть пути в институт или на работу и обратно идти пешком. Путь разделить на 3 части. Первая – втягивающая. Идти нужно не  торопясь, чтобы организм приспособился к ритму ходьбы. Вторая часть пути – основная составляет половину дистанции. Идти следует быстрее, но не на пределе возможностей «разговорного режима». Третья часть пути – успокаивающая. Сбавить темп и привести дыхание в норму. Нельзя допускать резких переходов от интенсивного движения к покою во избежание чрезмерной нагрузки на сердце. Когда ходьба на работу станет привычкой нужно завести дневник для записи пройденных километров, времени в пути, средней скорости, самочувствия, пульса, веса. Так контролируют степень  нагрузки. Тем, кто постоянно делает утреннюю зарядку и не имеет дефектов походки и осанки, первый этап можно начать с освоения утром специальных упражнений скорохода и проводить специальное занятие на местности один раз в неделю наряду с ходьбой по городу. Упражнения скорохода имитируют движения всех звеньев тела, совершаемые при ходьбе:

    \begin{itemize}
        \item В положении стоя боком к опоре делать махи вперед и назад дальней от опоры ногой с наибольшей амплитудой движений, но расслабленно. Туловище держать прямо. Усложнять, поставив опорную ногу на возвышение, что создаст условия для свободного провисания маховой ноги, как это происходит при ходьбе. Дальше усложнять с поворотом таза вокруг вертикальной оси. Темп махов – от спокойного к быстрому. Повторять упражнение 2-4 раза за одно занятие.
        \item В положении основной стойки делать круговые вращения тазом в обе стороны. Усложнять, раздвигая стопы на ширину плеч. Повторять 2-4 раза за 1 занятие.
        \item В положении основной стойки имитировать движения руками при ходьбе. Начинать махи почти прямыми руками, спокойно, расслабленно и с широкой амплитудой движений. Увеличивать темп, руки начнут сгибаться, как при быстрой ходьбе. Плечи во время движений не поднимать. Пальцы сжать в кулак, но не напрягать.
        \item В положении основной стойки вращать туловищем в обе стороны. Голову держать прямо. Усложнять, поднимая руки над головой.
        \item Стоя в положении широкого шага, проделать перекатывание на стопе с пятки на носок, одновременно осуществляя поворот туловища на 180º и обратно.
        \item В положении основной стойки делать круговые вращения коленями. Усложнять, наклонившись вперед и помогая коленям руками, что увеличивает амплитуду вращения.
        \item Из положения выпада одной ногой вперед делать пружинистые покачивания телом; сменив положение ног прыжком, повторить движения.
    \end{itemize}

    Комплекс специальных упражнений – непременная часть ежедневной  подготовки спортсменов-скороходов самого высокого класса… После них быстрее налаживается пластичный и гармоничный стиль ходьбы.

    Во время специальных занятий на местности по выходным дням  необходимо сначала размяться: 80-100 метров пройти в парке (в лесу) и  сделать 2-3 упражнения. Потом опять пройти 100-150 метров и сделать 3-4  упражнения и так 15 минут. К специальным упражнениям скорохода нужно добавить еще 5, подходящих для открытой местности:

    \begin{itemize}
        \item Ходьба восьмеркой с диаметром от 5 до 3 метров для отрабатывания поворотов таза при ходьбе;
        \item Метание легкого предмета. Камешек (сучок) поднять над головой, сделать 2-3 быстрых, широких шага, чтобы ноги и таз опередили плечи, и бросить предмет. Так вырабатывается навык энергичного отталкивания ногами от земли при ходьбе и улучшается осанка;
        \item Ходьба боком скрестными шагами укрепляет мышцы таза (шаг правой ногой перед (или сзади) левой и так далее, чередуя ноги);
        \item Ходьба семенящей походкой по тропинкам с мягким грунтом (газонам). Для этого нужно опустить расслабленно руки и плечи и идти быстрой семенящей походкой, стараясь до отказа разгибать ноги в коленях. Помогает соизмерять частоту шагов и полнее разгибать ноги в коленях при ходьбе;
        \item Ходьба выпадами с поворотом плеча в сторону разноименной ноги (при шаге с правой ноги левое плечо поворачивается вперед и наоборот). Упражнение укрепляет мышцы живота.
    \end{itemize}

    После разминки можно идти по маршруту с втягивающей, основной и  заключительной частью. Окончание – несколько упражнений скорохода для восстановления гибкости тела.

    Вторым этапом оздоровительной программы является этап повышения скорости ходьбы. Добиваться этого следует во время тренировок на местности: во время пешей прогулки основную часть нужно проводить в переменном темпе: 200-300 метров пройти быстро (с наивысшей скоростью). Потом сбавить скорость и восстановить силы для нового ускорения. Ускорение должно приносить удовольствие, а не забирать все силы. Гайс И.А. пишет, что повышение скорости ходьбы – не самоцель, а  лишь средство для достижения тренировочного эффекта, благотворного для сердечной деятельности, и приводит слова академика Н.М. Амосова: «Перетренировка – это  уже болезнь». К концу второго этапа организм должен быть не только готов поддерживать скорость ходьбы, но и перейти от 2 спецзанятий в неделю к 3 и более. Системы могут быть разные: 3-4-х разовые еженедельные спецзанятия по 1 часу каждое или ежедневные, но по 20-30 минут, или ежедневные по 1 часу и «ударные» (2 и более часов в выходные  дни). Главное, чтобы нагрузка  нарастала  постепенно и была по силам.

    На заключительном этапе занятий надо просто совершенствовать  приемы ходьбы и поддерживать необходимую для этого физическую форму. Желательно, чтобы занятия были ежедневными. Пусть они будут менее часа в будние дни, выполняя поддерживающую задачу. А в выходные  дни тренировка будет более насыщенной и продолжительной. Чтобы не «заело» однообразие специалисты советуют обновлять комплексы упражнений  разминки, включая в занятия игры без нагрузок и ударов по ногам (бадминтон, волейбол, гольф и т.п.), формировать традиции (прогулка перед сном, пеший переход в новогодний вечер и т.д.). При длительных переходах в выходные дни (более 2 часов) необходимо брать с собой воду, а иногда и питание. Потеря жидкости с потом при переходе в жару может доходить до 2,8 литров в час. А сердечно-сосудистая система должна иметь достаточный объем крови для доставки питательных веществ и кислорода к мышцам, а также удалять избытки тепла. В противном случае неизбежен тепловой удар. Поэтому  нужно соблюдать питьевой режим. До выхода в поход выпить 200-300 граммов воды, даже если нет жажды. При возникновении жажды во время перехода нужно пить. После первых полутора часов лучше выпить чистой воды, так как это активизирует обмен жиров в организме и способствует похудению.   Длительная нагрузка может привести к снижению уровня глюкозы в крови (состояние гипогликемии или «волчьего голода»). Это состояние наступает не ранее 2-3 часов после начала перехода. Поэтому через 2 часа после начала  ходьбы лучше выпить сладкий напиток (количество сахара не более 25 граммов на литр). Раствор слабой концентрации быстрее всасывается. А крепкий сахарный раствор препятствует возмещению водного дефицита, замедляя всасывание в слизистую оболочку желудка. Что касается дыхания, то это акт рефлекторный. Организм сам переходит на комплексное дыхание через рот и нос по мере возрастания скорости ходьбы.

\end{document}
