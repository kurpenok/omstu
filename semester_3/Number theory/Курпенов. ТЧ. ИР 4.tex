\documentclass[14pt, a4paper]{article}

\usepackage[T2A]{fontenc}
\usepackage[utf8]{inputenc}
\usepackage[english, russian]{babel}

\usepackage{amsmath}

\begin{document}
    \thispagestyle{empty}

    \begin{center}
        Министерство науки и высшего образования Российской Федерации

        Федеральное государственно автономное образовательное учреждение высшего образования

        <<Омский государственный технический университет>>

        \vspace{1cm}
        Факультет информационных технологий и компьютерных систем

        Кафедра <<Прикладная математика и фундаметральная информатика>>

        \vspace{3cm}
        \textbf{Индивидуальная работа}

        по дисциплине <<Теория чисел>>
    \end{center}
    
    \vspace{3cm}
    \begin{flushright}    
        \begin{tabular}{ r r }
            Студента & Курпенова Куата Ибраимовича \\
            \cline{2-2}
            & \tiny{фамилия, имя, отчество полностью} \\

            Курс & 2, группа ФИТ-212 \\
            \cline{2-2}
            Направление & 02.03.02 Прикладная математика \\
            \cline{2-2}
            & и фундаментальная информатика \\
            \cline{2-2}
            & \tiny{код, наименование} \\
            
            Руководитель & доц., канд. физ.-мат. наук \\
            \cline{2-2}
            & \tiny{должность, ученая степень, звание} \\
            & Белим С. Ю. \\
            \cline{2-2}
            & \tiny{фамилия, инициалы} \\
            
            Выполнил & \\
            \cline{2-2}
            & \tiny{дата, подпись студента} \\
            
        \end{tabular}
    \end{flushright}
    
    \vspace*{\fill}
    \begin{center}
        Омск 2022
    \end{center}

    \newpage

    \section*{Задание 1}
    Вычислите значение символа Лежандра: $\frac{111}{541}$.

    \subsection*{Решение}
    
    \[ \frac{111}{541} \equiv \frac{97}{111} \equiv \frac{14}{97} \equiv \frac{2 \cdot 7}{97} \equiv \frac{7}{97} \equiv \frac{6}{97} \equiv \frac{3}{7} \equiv -\frac{1}{3} \equiv -1 \]

    \subsection*{Ответ}
    Символ Лежандра равен -1.

    \vspace{2cm}
    \section*{Задание 2}
    Вычислите значение символа Лежандра с помощью критерия Эйлера: $\frac{11}{37}$.

    \subsection*{Решение}

    \[ \frac{11}{37} \equiv 11^{\frac{37 - 1}{2}}(mod37) \equiv 11^{12}(mod37) \]

    \[ 11^{12} \equiv 1(mod37) \]
    \[ 11^{12} - 1 \equiv 0(mod37) \]
    \[ (11^{6} + 1)(11^{6} - 1) \equiv 0(mod37) \]
    
    \[ 11^{2} \equiv 121(mod37) \equiv 10(mod37) \equiv -1(mod37) \]
    \[ (-1)^{3}mod(37) \equiv -1(mod(37) \]
    \[ 11^{12} \equiv 1(mod37) \]

    \[ 11^{12}(mod37) \equiv 313842837672(mod37) \equiv 1(mod37) \]

    \subsection*{Ответ}
    Символ Лежандра равен 1.

    \vspace{2cm}
    \section*{Задание 3}
    Решите сравнение: $x^{2} \equiv 5(mod29)$.

    \subsection*{Решение}

    Рассмотрим символ Лежандра, так как 29 - простое число.

    \[ \frac{5}{29} \equiv (-1)^{\frac{5-1}{2} \cdot \frac{29-1}{2}} \frac{29}{5} \equiv \frac{4}{5} \equiv (-1)^{\frac{4-1}{2} \cdot \frac{5-1}{2}}\frac{5}{4} \equiv \frac{1}{4} \equiv 1\]
    \[ 5 \in Q_{29} \]

    \[ 1 \equiv \frac{5}{29} \equiv 5^{\frac{29-1}{2}}(mod29) \]
    \[ 1 \equiv 5^{14}(mod29) \]
    \[ 5^{14} \equiv 1(mod29) \]
    \[ 5^{14} - 1 \equiv 0(mod29) \]
    \[ (5^{7} -1)(5^{7} + 1) \equiv 0(mod29)\]
    \[ 5^{3} \equiv 125(mod29) \equiv 9(mod29) \]
    \[ 5^{4} \equiv 625(mod29) \equiv 16(mod29) \]
    \[ 5^{7} \equiv 9 \cdot 16 \equiv 144(mod29) \equiv 28(mod29) \equiv -1(mod29) \]

    Возьмём квадратичный невычет по модулю 29:

    \[ \frac{2}{29} \equiv 2^{\frac{29-1}{2}}(mod29) \equiv 2^{14}(mod29) \]
    \[ -1 \equiv 2^{14}(mod29) \]
    \[ 2^{14} \cdot 5^{7} \equiv 1(mod29) \]
    \[ 2^{14} \cdot 5^{8} \equiv 5(mod29) \]
    \[ x \equiv \pm(5^{4} \cdot 2^{7})(mod29) \]

    \subsection*{Ответ}

    \begin{equation*}
        x \equiv 
        \begin{cases}
            11(mod29) \\
            18(mod29) \\
        \end{cases}
    \end{equation*}
\end{document}
