\documentclass[14pt, a4paper]{article}

% Language setup
\usepackage[T2A]{fontenc}
\usepackage[utf8]{inputenc}
\usepackage[english, russian]{babel}

% Math modules setup
\usepackage{asmath}

\begin{document}
    \section*{Конспекты лекций по машинному обучению}

    \subsection*{Формализация}

    $x$ --- объект.
    $X$ --- множество объектов.
    $y = y(x)$ --- ответ (метки).
    $Y$ --- множество ответов.

    $a: \X \rightarrow  \Y$ --- алгоритм, модель.
    $a(x) = y`$ --- прогноз.
    $a(x; w)$, где $w$ --- параметры весы. Пример: $a(x, w_0, w_1) = w_1 x + w_0$
    $\A$ --- семейство алгоритмом $a$.


    \subsection*{Цель машинного обучения}

    Задача --- найти наиболее точный $y`$ к $y$.

    Схема машинного обучения:
    \begin{itemize}
        \item Входные данные $\X$
        \item Алгоритм
        \item Предсказание $Y`$
        \item Истинное значение $\Y$
        \item Функция потерь
        \item Оценка потерь (Если оценка не устраивает --- следует поменять класс обучаемой модели)
        \item Оптимизатор (корректировка весов) --- гарантированно отдаёт решение не хуже предыдущего
        \item Веса (goto: Алгоритм)
    \end{itemize}

    Схема рабочего процесса система с машинным обучением:
    \begin{itemize}
        \item Новые данные
        \item Модель прогнозирования
        \begin{itemize}
            \item Метки
            \item Тренировочные данные
            \item Повторное обучение алгоритма прогнозирования
        \end{itemize}
        \item Прогноз
    \end{itemize}


    \subsection*{Функция потерь (loss function)}

    \[ L(y(x) - y`(x)) = L(y(x) - a(x)) \]

    \[ Q(a, \Xi) = \sum{x \in \Xi} L(y(x) - a(x)) \]

    $Q(a, \Xi)$ --- ошибка алгоритма $a$ на выборке $\Xi$

    Обучение: $Q(a, \Xi) \rightarrow min$

    Какие вопросы стоят перед инженером data-science:
    \begin{itemize}
        \item Как получить данные?
        \item Как подготовить данные?
        \item Как сформировать признаки? ($\Xi_j, j = \overline{1, d}$)
        \item Каким выбрать функционал ошибки? (Какую функцию потерь выбрать?) ($Q$)
        \item Какое взять семейство алгоритмов? ($\A$)
        \item Как обучить алгоритм? (Какой метод обучения использовать?)
    \end{itemize}

    
    \subsection*{Искусственный интеллект}
    
    Искусственный интеллект --- это отрасль информатики, целью которой является создание интеллектуальных машин, способных выполнять задачи, обычно требующие человеческого интеллекта, такие как визуальное восприятие, распознавание речи, принятие решений и перевод между языками. ИИ добился значительных успехов в последние годы и быстро становится неотъемлемой частью повседневной жизни.

    Автономность --- это свойство системы, позволяющее ей действовать и, главное, принимать решения самостоятельно, без дополнительных управляющих воздействий извне.

    Адаптивность --- это свойство системы, которые позволяет ей действовать в условиях изменчивости внешних воздействий.

    Формула ИИ: $BigData + возможности быстрой итеративной обратобки + интеллектуальные алгоритмы$

    Типы искусственного интеллекта:
    \begin{itemize}
        \item Слабый --- автоматизация рутинной работы и подражание человеку
        \item Сильный --- обладает творческие способности, дублирование человеческого интеллекта
    \end{itemize}
\end{document}
