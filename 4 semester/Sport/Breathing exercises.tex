\documentclass[14pt, a4paper]{article}

\usepackage[T2A]{fontenc}
\usepackage[utf8]{inputenc}
\usepackage[english, russian]{babel}

\begin{document}
    \thispagestyle{empty}

    \begin{center}
        Министерство науки и высшего образования Российской Федерации

        Федеральное государственно автономное образовательное учреждение высшего образования

        <<Омский государственный технический университет>>

        \vspace{1cm}
        Факультет информационных технологий и компьютерных систем

        Кафедра <<Прикладная математика и фундаметральная информатика>>

        \vspace{3cm}
        \textbf{Лабораторная работа}

        по дисциплине <<Физическая культура>>
    \end{center}
    
    \vspace{3cm}
    \begin{flushright}    
        \begin{tabular}{ r r }
            Студента & Курпенова Куата Ибраимовича \\
            \cline{2-2}
            & \tiny{фамилия, имя, отчество полностью} \\

            Курс & 2, группа ФИТ-212 \\
            \cline{2-2}
            Направление & 02.03.02 Прикладная математика \\
            \cline{2-2}
            & и фундаментальная информатика \\
            \cline{2-2}
            & \tiny{код, наименование} \\
            
            Руководитель & доц., канд. пед. наук \\
            \cline{2-2}
            & \tiny{должность, ученая степень, звание} \\
            & Сафонова Ж. Б. \\
            \cline{2-2}
            & \tiny{фамилия, инициалы} \\
            
            Выполнил & \\
            \cline{2-2}
            & \tiny{дата, подпись студента} \\
            
        \end{tabular}
    \end{flushright}
    
    \vspace*{\fill}
    \begin{center}
        Омск 2023
    \end{center}

    \newpage

    \tableofcontents

    \newpage

    \section{Дыхательная гимнастика}

    \subsection{Введение}
    Дыхание - это жизнь. Справедливость такого утверждения вряд ли у кого-нибудь вызовет возражение. Действительно, если без твердой пищи организм может обходиться несколько месяцев, без воды - несколько дней, то без воздуха- всего несколько минут.

    Приоритетность процесса дыхания для жизни делает способность в совершенстве владеть этим процессом едва ли не главной способностью человека творить чудеса со своим организмом, избавляться от болезней, становиться здоровым. Это уже давно доказали индийские йоги, которые могут обходиться без дыхания значительно дольше, чем обычные люди.

    С помощью дыхания можно вводить организм в состояние возбуждения (как это делается в боевых искусствах Востока) и максимального расслабления (йоги способны вводить себя в состояние клинической смерти).

    Существует много разновидностей дыхательной гимнастики. В настоящее время наиболее популярными являются: парадоксальное дыхание по А.Н. Стрельниковой, поверхностное дыхание по К.П. Бутейко, редкое и глубокое дыхание по системе йогов, метод Фролова (тренажер Фролова).

    Целью данной работы является рассмотрение дыхательной гимнастики по методу А.Н. Стрельниковой и К.П. Бутейко. Так они являются самыми наиболее доступными и эффективными из всех видов дыхательной гимнастики.

    \subsection{Дыхательная гимнастика по методу А. Н. Стрельниковой}
    Из четырех функций органов дыхания: дышать, говорить, кричать и петь -пение самая сложная. Следовательно, гимнастика, которая восстанавливает даже певческий голос, то есть, самую сложную функцию, по дороге к цели неизбежно восстанавливает функции более простые, и прежде всего дыхание.

    Гимнастика А.Н. Стрельниковой - единственная в мире, в которой короткий и резкий вдох носом делается на движениях, сжимающих грудную клетку.

    Упражнения активно включают в работу все части тела (руки, ноги, голову, бедерный пояс, брюшной пресс, плечевой пояс и т.д.) и вызывают общую физиологическую реакцию всего организма, повышенную потребность в кислороде. Так как все упражнения выполняются одновременно с коротким и резким вдохом через нос (при абсолютно пассивном выдохе), это усиливает внутреннее тканевое дыхание и повышает усвояемость кислорода тканями, а также раздражает ту обширную зону рецепторов на слизистой оболочке носа, которая обеспечивает рефлекторную связь полости носа почти со всеми органами. Вот почему эта дыхательная гимнастика имеет такой необыкновенно широкий спектр воздействия, помогает при массе различных заболеваний органов и систем.

    \subsection{Лечебное воздействие гимнастики}
    Если систематически выполнять стрельниковскую дыхательную гимнастику (два раза в день- утром и вечером, по 1200 вдохов-движений за одно занятие), то результаты не замедлят сказаться.

    Стрельниковская гимнастика оказывает на организм человека комплексное воздействие:
    \begin{itemize}
        \item положительно влияет на обменные процессы, играющие важную роль в кровоснабжении, в том числе и легочной ткани;
        \item способствует восстановлению нарушенных в ходе болезни нервных регуляций со стороны центральной нервной системы;
        \item улучшает дренажную функцию бронхов;
        \item восстанавливает нарушенное носовое дыхание;
        \item устраняет некоторые морфологические изменения в бронхолегочной системе (спайки, слипчатые процессы);
        \item способствует рассасыванию воспалительных образований, расправлению воспалительных образований, расправлению сморщенных участков легочной ткани, восстановлению нормального крово- и лимфоснабжения, устранению местных застойных явлений;
        \item налаживает нарушенные функции сердечно-сосудистой системы, укрепляет весь аппарат кровообращения;
        \item исправляет развивающиеся в процессе заболевания различные деформации грудной клетки и позвоночника;
        \item повышает общую сопротивляемость организма, его тонус, оздоровляет нервно-психическое состояние у больных.
    \end{itemize}

    \subsection{Преимущества гимнастики}
    \begin{itemize}
        \item Гимнастика сочетается со всеми циклическими упражнениями: ходьба, бег, плавание - особенно.
        \item Там, где болезнь, там, где она "сидит" в вас, гимнастика восстанавливает функции, разрушенные болезнью.
        \item Дыхательная гимнастика - отличная профилактика болезней.
        \item Гимнастика положительно влияет на организм в целом. Она ведь не чисто дыхательная - в работу включаются все мышцы.
        \item Гимнастика доступна всем людям.
        \item Для занятий гимнастикой не требуется особых условий – специальной одежды (спортивный костюм, кроссовки и т.д.), помещения и прочих.
        \item Высокая эффективность. После первых занятий объем легких значительно увеличивается.
        \item Дает хороший эффект для тренировки мышечной системы дыхательного аппарата и грудной клетке.
        \item Гимнастика показана и взрослым, и детям.
    \end{itemize}

    \subsection{Основные правила выполнения}
    \begin{itemize}
        \item Думайте только о вдохе носом, тренируйте только вдох. Вдох - шумный, резкий и короткий (как хлопок в ладоши).
        \item Выдох должен уходить после каждого вдоха самостоятельно через рот. Не задерживайте и не выталкивайте выдох. Вдох - предельно активный (носом), выдох - абсолютно пассивный (через рот).
        \item Вдох делается одновременно с движением. В Стрельниковской дыхательной гимнастике нет вдоха без движения. А движения - без вдоха.
        \item Все вдохи-движения Стрельниковской гимнастики делаются в темпо ритме строевого шага.
        \item Счет в Стрельниковской дыхательной гимнастике только на 8, считать мысленно, не вслух.
        \item Упражнения можно делать стоя, сидя, лежа.
    \end{itemize}

    \textbf{Правило 1}
    "Гарью пахнет! Тревога!" И резко, шумно, на всю квартиру, нюхайте воздух, как собака след. Чем естественнее, тем лучше.

    Самая грубая ошибка - тянуть воздух, чтобы взять воздуха побольше. Вдох короткий, как укол, активный и чем естественнее, тем лучше. Думайте только о вдохе. Чувство тревоги организует активный вдох лучше, чем рассуждения о нем. Поэтому, не стесняясь, яростно, до грубости, нюхайте воздух.

    \textbf{Правило 2}
    Выдох - результат вдоха.

    Не мешайте выдоху уходить после каждого вдоха как угодно, сколько угодно- но лучше ртом, чем носом. Не помогайте ему. Думайте только: "Гарью пахнет!

    Тревога!" И следите за тем только, чтобы вдох шел одновременно с движением.

    Выдох уйдет самопроизвольно. Во время гимнастики рот должен быть слегка приоткрыт. Увлекайтесь вдохом и движением, не будьте скучно-равнодушными.

    Играйте в дикаря, как играют дети, и все получится. Движения создают короткому вдоху достаточный объем и глубину без особых усилий.

    \textbf{Правило 3}
    Повторяйте вдохи так, как будто вы накачиваете шину в темпо ритме песен и плясок. И, тренируя движения и вдохи, считайте на 2, 4 и 8. Темп - 60-72 вдоха в минуту. Вдохи громче выдохов. Норма урока - 1000-1200 вдохов, можно и больше - 2000 вдохов. Паузы между дозами вдохов - 1-3 секунды.

    \textbf{Правило 4}
    Подряд делайте столько вдохов, сколько в данный момент можете сделать легко.

    Весь комплекс состоит из 8 упражнений. В начале - разминка. Встаньте прямо.

    Руки по швам. Ноги на ширине плеч. Делайте короткие, как укол, вдохи громко, шмыгая носом. Не стесняйтесь. Заставьте крылья носа соединяться в момент вдоха, а не расширяйте их. Тренируйте по 2, по 4 вдоха подряд в темпе прогулочного шага "сотню" вдохов. Можно и больше, чтобы ощутить, что ноздри двигаются и слушаются вас. Вдох, как укол, мгновенный. Думайте:

    "Гарью пахнет! Откуда?"

    Чтобы понять нашу гимнастику, делайте шаг на месте и одновременно с каждым шагом - вдох. Правой-левой, правой-левой, вдох-вдох, вдох-вдох. А не вдох-выдох, как в обычной гимнастике.

    Сделайте 96 (сотню) шагов-вдохов в прогулочном темпе. Можно, стоя на месте, можно при ходьбе по комнате, можно переминаясь с ноги на ногу: вперед-назад, вперед-назад, тяжесть тела то на ноге, стоящей впереди, то на ноге, стоящей сзади. В темпе шагов делать длинные вдохи невозможно. Думайте: «ноги накачивают в меня воздух". Это помогает. С каждым шагом - вдох, короткий, как укол, и шумный.

    Освоив движение, поднимая правую ногу, чуть-чуть приседайте на левой, поднимая левую - на правой. Получится танец рок-н-ролл. Следите за тем, чтобы движения и вдохи шли одновременно. Не мешайте и не помогайте выходить выдохам после каждого вдоха. Повторяйте вдохи ритмично и частою. Делайте их столько, сколько сможете сделать легко.

    \subsection{Заключение}
    Таким образом, всё выше сказанное позволяет сделать вывод о том, что различные дыхательные гимнастики в последние годы уверенно входят в терапевтическую практику. Они помогают многим больным. Методика этих гимнастик заключается в особенном дыхании. Так в гимнастике по А.Н. Стрельниковой - короткий и резкий вдох носом делается на движениях, сжимающих грудную клетку. А в методе К.П. Бутейко универсальный комплекс дыхательных упражнений, направленный на развитие поверхностного, глубокого, редкого дыхания, а также на развитие способности человека задерживать дыхание как на вдохе, так и на выдохе, как в состоянии покоя, так и при физической нагрузке. В любом случае, эти виды гимнастик имеют ряд преимуществ. И все они основаны на носовом дыхании. Не случайно поэтому, йоги предупреждают: если дети не будут дышать через нос, то не получат достаточно умственного развития, т.к. носовое дыхание стимулирует нервные окончания всех органов, находящиеся в носоглотке.

    И наша задача научиться правильно дышать!
\end{document}
