\documentclass[14pt, a4paper]{article}

\usepackage[T2A]{fontenc}
\usepackage[utf8]{inputenc}
\usepackage[english, russian]{babel}

\usepackage{amsmath}

\begin{document}
    \thispagestyle{empty}

    \begin{center}
        Министерство науки и высшего образования Российской Федерации

        Федеральное государственно автономное образовательное учреждение высшего образования

        <<Омский государственный технический университет>>

        \vspace{1cm}
        Факультет информационных технологий и компьютерных систем

        Кафедра <<Прикладная математика и фундаметральная информатика>>

        \vspace{3cm}
        \textbf{Индивидуальная работа}

        по дисциплине <<Математическая логика и теория алгоритмов>>
    \end{center}
    
    \vspace{3cm}
    \begin{flushright}    
        \begin{tabular}{ r r }
            Студента & Курпенова Куата Ибраимовича \\
            \cline{2-2}
            & \tiny{фамилия, имя, отчество полностью} \\

            Курс & 2, группа ФИТ-212 \\
            \cline{2-2}
            Направление & 02.03.02 Прикладная математика \\
            \cline{2-2}
            & и фундаментальная информатика \\
            \cline{2-2}
            & \tiny{код, наименование} \\
            
            Руководитель & доц., канд. физ.-мат. наук \\
            \cline{2-2}
            & \tiny{должность, ученая степень, звание} \\
            & Белим С. Ю. \\
            \cline{2-2}
            & \tiny{фамилия, инициалы} \\
            
            Выполнил & \\
            \cline{2-2}
            & \tiny{дата, подпись студента} \\
            
        \end{tabular}
    \end{flushright}
    
    \vspace*{\fill}
    \begin{center}
        Омск 2022
    \end{center}

    \newpage

    \section*{Задание 1}
    Используя метод резолюций с унификацией предикатных выражений для заданного множества гипотез $\{F_1, F_2, \dots, F_n\}$ и утверждения $B$, доказать справедливость $F_1, F_2, \dots, F_n \vdash B$.

    \subsection*{Решение}

    \[ F_1 = \forall{x}(K(x) \& \forall{y}(R(y) \rightarrow U(x, y))) \]
    \[ F_1 = \forall{x}\forall{y}(K(x) \& (\neg R(y) \vee U(x, y))) \]
    \[ F_1 = \forall{x}\forall{y}((K(x) \& \neg R(y)) \vee (K(x) \& U(x, y))) \]

    \[ F_2 = \forall{x}(K(x) \& \forall{y}(B(y) \rightarrow \neg U(x, y))) \]
    \[ F_2 = \forall{x}\forall{y}(K(x) \& (\neg B(y) \vee \neg U(x, y))) \]
    \[ F_2 = \forall{x}\forall{y}((K(x) \& \neg B(y)) \vee (K(x) \& \neg U(x, y))) \]

    \[ B = \forall{y}(R(y) \rightarrow \neg B(y)) = \forall{y}(\neg R(y) \vee \neg B(y))\]
    \[ \neg B = \neg (\forall{y}(\neg R(y) \vee \neg B(y))) = \exists{y}(R(y) \& B(y)) = R(c) \& B(c) \]

    \begin{align*}
        K: \{
            & (K(x) \& \neg R(y)) \vee (K(x) \& U(x, y)), \\
            & (K(x) \& \neg B(y)) \vee (K(x) \& \neg U(x, y)), \\
            & R(c), B(c) \} \\
    \end{align*}

    \[ res((K(x) \& \neg R(y)) \vee (K(x) \& U(x, y)), R(c)) = K(x) \& R(c) \& U(x, y) \]
    \[ res(K(x) \& R(c) \& U(x, y), (K(x) \& \neg B(y)) \vee (K(x) \& \neg U(x, y))) = K(x) \& R(c) \& \neg B(y) \]
    \[ res(K(x) \& R(c) \& \neg B(y), B(c)) = False\]

    \subsection*{Ответ}
    Справедливость доказана.

    \vspace{1cm}
    \section*{Задание 2}
    Дана машина Тьюринга с внешним алфавитом $A = \{a_0, 1\}$, алфавитом внутренних состояний $Q = \{q_0, q_1, q_2, q_3, q_4, q_5, q_6, q_7, R, L, S\}$ и со следующей программой:

    \begin{center}
        \begin{tabular}{ | c | c | c | c | c | c | c | c | } 
            \hline
            & $q_1$ & $q_2$ & $q_3$ & $q_4$ & $q_5$ & $q_6$ & $q_7$ \\
            \hline
            $a_0$ & $q_4 a_0 R$ & $q_6 a_0 R$ & $q_6 a_0 R$ & $q_0 1 R$ & $q_4 a_0 R$ & $q_0 a_0 S$ & $q_6 a_0 R$ \\
            \hline
            $1$ & $q_2 1 L$ & $q_3 1 L$ & $q_1 1 L$ & $q_5 a_0 S$ & $q_5 a_0 S$ & $q_7 a_0 S$ & $q_7 a_0 R$ \\
            \hline
        \end{tabular}
    \end{center}
    Стартовое состояние:
    
    \begin{center}
        \begin{tabular}{ | c c c c c | }
            \hline
            1 & 1 & 1 & 1 & 1 \\
            & & & & $q_1$ \\
            \hline
        \end{tabular}
    \end{center}

    \subsection*{Решение}

    \begin{center}
        \begin{tabular}{ | c c c c c | }
            \hline
            1 & 1 & 1 & 1 & 1 \\
            & & & & $q_1$ \\
            \hline
        \end{tabular}
    \end{center}
    
    \begin{center}
        \begin{tabular}{ | c c c c c | }
            \hline
            1 & 1 & 1 & 1 & 1 \\
            & & & $q_2$ & \\
            \hline
        \end{tabular}
    \end{center}
    
    \begin{center}
        \begin{tabular}{ | c c c c c | }
            \hline
            1 & 1 & 1 & 1 & 1 \\
            & & $q_3$ & & \\
            \hline
        \end{tabular}
    \end{center}
    
    \begin{center}
        \begin{tabular}{ | c c c c c | }
            \hline
            1 & 1 & 1 & 1 & 1 \\
            & $q_1$ & & & \\
            \hline
        \end{tabular}
    \end{center}
    
    \begin{center}
        \begin{tabular}{ | c c c c c | }
            \hline
            1 & 1 & 1 & 1 & 1 \\
            $q_2$ & & & & \\
            \hline
        \end{tabular}
    \end{center}
    
    \begin{center}
        \begin{tabular}{ | c c c c c c | }
            \hline
            $a_0$ & 1 & 1 & 1 & 1 & 1 \\
            $q_3$ & & & & & \\
            \hline
        \end{tabular}
    \end{center}
    
    \begin{center}
        \begin{tabular}{ | c c c c c | }
            \hline
            1 & 1 & 1 & 1 & 1 \\
            $q_6$ & & & & \\
            \hline
        \end{tabular}
    \end{center}
    
    \begin{center}
        \begin{tabular}{ | c c c c c | }
            \hline
            $a_0$ & 1 & 1 & 1 & 1 \\
            $q_7$ & & & & \\
            \hline
        \end{tabular}
    \end{center}
    
    \begin{center}
        \begin{tabular}{ | c c c c c | }
            \hline
            $a_0$ & 1 & 1 & 1 & 1 \\
            & $q_6$ & & &  \\
            \hline
        \end{tabular}
    \end{center}
    
    \begin{center}
        \begin{tabular}{ | c c c c | }
            \hline
            $a_0$ & 1 & 1 & 1 \\
            $q_7$ & & & \\
            \hline
        \end{tabular}
    \end{center}
    
    \begin{center}
        \begin{tabular}{ | c c c c | }
            \hline
            $a_0$ & 1 & 1 & 1 \\
            & $q_6$ & &   \\
            \hline
        \end{tabular}
    \end{center}
    
    \begin{center}
        \begin{tabular}{ | c c c | }
            \hline
            $a_0$ & 1 & 1 \\
            $q_7$ & & \\
            \hline
        \end{tabular}
    \end{center}
    
    \begin{center}
        \begin{tabular}{ | c c c | }
            \hline
            $a_0$ & 1 & 1 \\
            & $q_6$ & \\
            \hline
        \end{tabular}
    \end{center}
    
    \begin{center}
        \begin{tabular}{ | c c | }
            \hline
            $a_0$ & 1 \\
            $q_7$ & \\
            \hline
        \end{tabular}
    \end{center}
    
    \begin{center}
        \begin{tabular}{ | c c | }
            \hline
            $a_0$ & 1 \\
            & $q_6$ \\
            \hline
        \end{tabular}
    \end{center}
    
    \begin{center}
        \begin{tabular}{ | c | }
            \hline
            $a_0$ \\
            $q_7$ \\
            \hline
        \end{tabular}
    \end{center}
    
    \begin{center}
        \begin{tabular}{ | c c | }
            \hline
            $a_0$ & $a_0$ \\
            & $q_6$ \\
            \hline
        \end{tabular}
    \end{center}
    
    \begin{center}
        \begin{tabular}{ | c | }
            \hline
            $a_0$ \\
            $q_0$ \\
            \hline
        \end{tabular}
    \end{center}

    \subsection*{Ответ}
    Данная программа удаляет число с ленты.

    \vspace{1cm}
    \section*{Задание 3}
    Дана машина Тьюринга с внешним алфавитом $A = \{a_0, 1, 2, 3, 4, 5, 6, 7, 8, 9\}$. На ленте запись неотрицательного целого числа в десятичной системе. Стартовая конфигурация - головка под первой цифрой числа. Требуется получить на ленте запись числа, которое на 1 больше заданного числа (состояний головки не более трёх).

    \subsection*{Решение}
    
    \begin{center}
        \begin{tabular}{ | c | c | c | c | c | c | c | c | c | c | c | c | } 
            \hline
            & $a_0$ & 0 & 1 & 2 & 3 & 4 & 5 & 6 & 7 & 8 & 9 \\
            \hline
            $q_1$ & $q_2 a_0 L$ & $q_1 0 R$ & $q_1 1 R$ & $q_1 2 R$ & $q_1 3 R$ & $q_1 4 R$ & $q_1 5 R$ & $q_1 6 R$ & $q_1 7 R$ & $q_1 8 R$ & $q_1 9 R$ \\
            \hline
            $q_2$ & $q_0 1 S$ & $q_0 1 S$ & $q_0 2 S$ & $q_0 3 S$ & $q_0 4 S$ & $q_0 5 S$ & $q_0 6 S$ & $q_0 7 S$ & $q_0 8 S$ & $q_0 9 S$ & $q_2 0 L$ \\
            \hline
        \end{tabular}
    \end{center}

    \subsection*{Ответ}
    Прогамма для машины Тьюринга приведена выше.
\end{document}
