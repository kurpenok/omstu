\unnumsection{ВВЕДЕНИЕ}
\addcontentsline{toc}{section}{ВВЕДЕНИЕ}

Так как целью более масштабного проекта (робота-уборщика) является построение такого робота, который сможет самостоятельно передвигаться - требуется распознавать препятствия до объектов и определять расстояния до них. Эту проблему решает лидар - устройство, использующее невидимые для глаза лучи ИК-спектра для определения расстояний путём отражения лучей от объектов. Однако у данного решения есть большая проблема - он имеет существенную цену и плохо работает при плохих погодных условиях, например, в дождь или снег.

Цель проекта - создать устройство для определения расстояний, которое по своим характеристикам будет сравнимо с лидаром и будет лишено его главного недостатка - цены.

Для достижения поставленной цели требуется выполнить шаги из следующих задач:
\begin{itemize}
    \item Изучить техническую литературу из данной области (законы оптики, геометрия оптики, стереозрение)
    \item Приобрести стереокамеру для проеведения опытов
    \item Реализовать построение карты глубины на устройстве
\end{itemize}
