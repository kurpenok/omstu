\unnumsection{ЗАКЛЮЧЕНИЕ}
\addcontentsline{toc}{section}{ЗАКЛЮЧЕНИЕ}

В ходе выполнения курсового проекта были успешно изучены основы эпиполярной геометрии, что позволило глубже понять принципы работы стереокамер и их применения в задачах компьютерного зрения. Я проанализировал ключевые концепции, такие как эпиполя, фундаментальная матрица и соответствие точек, что стало основой для реализации нашего устройства.

Разработка системы на основе стереокамеры для построения карты глубины продемонстрировала эффективность применения полученных знаний на практике. В результате были реализованы алгоритмы, позволяющие точно определять расстояние до объектов в сцене и визуализировать трехмерную структуру окружающей среды. Проведенные эксперименты подтвердили работоспособность системы и её способность генерировать качественные карты глубины в реальном времени.

Полученные результаты открывают перспективы для дальнейших исследований и усовершенствований, таких как улучшение алгоритмов обработки изображений, интеграция с другими сенсорами и применение в различных областях, включая робототехнику, автономные системы и дополненную реальность. Я уверен, что данный проект стал важным шагом в моём понимании стереозрения и его практических приложений, и надеюсь на дальнейшее развитие в этой захватывающей области.
