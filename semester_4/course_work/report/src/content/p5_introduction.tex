\unnumsection{ВВЕДЕНИЕ}
\addcontentsline{toc}{section}{ВВЕДЕНИЕ}

Современные технологии трёхмерного восприятия окружающего мира находят все более широкое применение в различных областях, включая робототехнику, автономные транспортные средства, системы дополненной реальности, а также в области картографии и геодезии. Одним из ключевых аспектов этих технологий является способность точно измерять расстояния и строить карты глубины, что позволяет создавать детализированные трёхмерные модели окружающей среды.

Традиционно для получения данных о глубине используются лидары, которые обеспечивают высокую точность и качество измерений. Однако, несмотря на свои достоинства, такие системы имеют значительные недостатки, включая высокую стоимость, сложность в эксплуатации и чувствительность к погодным условиям. В результате этого, существует потребность в разработке более доступных и универсальных решений, которые могли бы предоставить сопоставимые результаты по более низкой цене.

В данной курсовой работе рассматривается разработка устройства на основе стереокамеры, которое способно строить карты глубины и выступает в роли недорогого аналога лидара. Стереозрение, как метод получения информации о глубине, основывается на анализе двух изображений, полученных с различных точек зрения, что позволяет вычислять расстояния до объектов в сцене. Этот подход не только снижает стоимость системы, но и делает её более доступной для широкого круга пользователей.

Важность данной работы заключается не только в создании нового устройства, но и в возможности его применения в различных сферах, включая автоматизацию процессов, улучшение навигации автономных устройств и восприятия ими окружающей среды, а также в образовательных проектах.

Таким образом, данная курсовая работа направлена на решение актуальной задачи создания недорогого и эффективного устройства для построения карт глубины, что имеет значительное значение в условиях современного технологического прогресса.
