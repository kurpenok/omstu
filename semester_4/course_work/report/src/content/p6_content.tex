\section{Постановка задачи}

В данной курсовой работе ставится задача разработки устройства на основе стереокамеры, способного строить карту глубины, аналогичную по функциональности лидарам, но с более доступной ценой. Основные цели и задачи, которые необходимо решить в рамках данной работы, включают:

\begin{enumerate}
    \item Получение фундаментальных знаний в области эпиполярной геометрии: найти статьи, учебники и другие материалы для большего погружения в тему стереоскопического зрения.

    \item Исследование существующего решения на основе лидаров: провести анализ методов построения карт глубины. Определить их преимущества и недостатки, а также выявить возможности для улучшения.

    \item Разработка архитектуры устройства: создать концепцию устройства, установить аппаратные и программные требования для достижения необходимой точности и производительности.

    \item Имплементация алгоритмов обработки: реализовать алгоритмы для обработки данных, получаемых от стереокамеры, включая калибровку, сопоставление изображений и построение карты глубины.

    \item Тестирование и валидация: провести испытания устройства. Сравнить полученные результаты с данными, полученными с помощью лидаров.

    \item Анализ результатов: оценить эффективность разработанного устройства, выявить его преимущества и ограничения, а также предложить рекомендации по дальнейшему улучшению и возможным направлениям применения.
\end{enumerate}




\newpage




\section{Решение}

\subsection{Описание математического аппарата стереозрения}

Для написания эффективного программного кода требуются фундаментальные знания в области решения задачи. В моём случае такой областью является \textit{эпиполярная геометрия}.

Эпиполярная геометрия это ключевая концепция в области стереозрения и компьютерного зрения, которая описывает взаимосвязи между трехмерными точками и их проекциями на двухмерные изображения, полученные с разных ракурсов. Эта геометрия позволяет ограничить пространство поиска соответствий между точками на изображениях, что значительно упрощает задачи, связанные с восстановлением трехмерной сцены. Приведём основные понятия эпиполярной геометрии:
\begin{enumerate}
    \item Эпиполь — это точка на плоскости изображения, где проецируется центр камеры другой плоскости изображения. В стереозрении есть два эпиполя: $e1$ и $e2$, которые соответствуют центрам камер $C1$ и $C2$ соответственно

    \item Базовая линия — это прямая, соединяющая центры двух камер $C1$ и $C2$. Она играет ключевую роль в определении эпиполярной плоскости.

    \item Эпиполярная плоскость — создаётся при пересечении базовой линии с плоскостями изображений обеих камер. Эта плоскость содержит точку 3D объекта и центры проекции обеих камер.

    \item Эпиполярная линия — это линия на плоскости изображения, которая соответствует проекции точки 3D объекта на другую камеру. Каждая точка на одной плоскости изображения соответствует линии на другой плоскости, что упрощает задачу поиска соответствий между изображениями.
\end{enumerate}

\image{epipolar_geometry.png}{Изображение, поясняющее эпиполярную геометрию}{0.5}

У нас есть эпиполярная плоскость $P$, созданная с использованием базовой линии $B$ и луча $R_1$. $e_1$ и $e_2$ — эпиполи, а $L_2$ — эпиполярная линия. Исходя из эпиполярной геометрии данного рисунка, пространство поиска для пикселя в изображении $i_2$, соответствующего пикселю $x_1$, ограничено одной двумерной линией, которая является эпиполярной линией для $i_2$. Это называется эпиполярным ограничением.

Есть ли способ представить всю эпиполярную геометрию единой матрицей? Более того, можем ли мы рассчитать эту матрицу, используя только два захваченных изображения? Хорошая новость в том, что такая матрица существует называется фундаментальной матрицей. Не вдаваясь в подробности, привожу формулу для расчёта фундаментальной матрицы:

\[x_2^T F x_1 = 0\]

Как только $F$ известна, мы можем найти эпиполярную линию $L_2$, используя формулу:

\[L_2 = F x_1\]

Если мы знаем $L_2$, то можно ограничить наш поиск пикселем $x_2$, соответствующим пикселю $x_1$, с помощью эпиполярного ограничения.




\subsection{Исследование существующего решения на основе лидаров}

Лидар (LiDAR, от английского "Light Detection and Ranging") — это технология, использующие лазерные лучи для определения расстояний до объектов и создания трёхмерных карт окружающей среды. Лидары работают по принципу измерения времени, необходимого лазерному лучу для отражения от объекта и возвращения к сенсору. Это время преобразуется в расстояние с использованием скорости света. Лидары могут генерировать до миллиона импульсов в секунду, создавая облака точек, которые затем обрабатываются для формирования детализированных 3D-моделей. Разберём преимущества и недостатки лидаров:

\begin{itemize}
    \item Высокая точность — лидары обеспечивают измерения с точностью до нескольких миллиметров, что делает их идеальными для задач, требующих высокой детализации.

    \item Широкий угол обзора — некоторые системы могут охватывать до 360 градусов, что позволяет получать полную картину окружающей среды.

    \item Низкие затраты на большие площади — лидары позволяют быстро и экономично собирать данные на больших территориях.

    \item Ограниченное восприятие — лидары не могут распознавать дорожные знаки или цвета светофоров, что может ограничивать их использование в некоторых сценариях.

    \item Высокие технические требования — для оптимальной работы требуется соблюдение определенных условий (например, температуры), а также регулярное техническое обслуживание.

    \item Чувствительность к погодным условиям — хотя лидары работают в сложных условиях, сильные дожди могут негативно влиять на их производительность.
\end{itemize}

\image{yandex_lidar.jpg}{Лидар, закреплённый на беспилотном автомобиле}{0.75}
\image{lidar_depthmap.jpg}{Карта глубины, полученная с лидара}{0.75}




\subsection{Описание устройства}

Построение устройства будет происходить на основе стереокамеры:

\image{stereocamera}{Внешний вид стереокамеры}{0.5}

К сожалению, фотографию и название личной стереокамеры прикрепить не имею возможности, так как брал камеру в аренду и на момент написания курсовой работу уже вернул устройство владельцу.




\subsection{Имплементация алгоритмов обработки}

В стереосистемах важно, чтобы обе камеры были правильно синхронизированы. Калибровка обеспечивает правильное определение взаимного расположения камер, что позволяет точно сопоставлять изображения и получать корректные эпиполярные линии, поэтому перед получением разницы между изображениями, нужно откалибровать устройство:

\image{calibrating.jpg}{Калибровка стереокамеры}{0.5}

После калибровки остаётся подобрать параметры для корректной работы библиотеки стереозрения:

\image{parameters.png}{Подбор параметров для получения корректной карты глубины}{0.5}




\subsection{Тестирование и валидация}

Так выглядит запущенное приложение для построения карты глубины:

\image{stereocamera_depthmap.png}{Результат работы устройства}{0.75}




\subsection{Анализ результатов}

Решение на моей стереокамере вполне позволяет различать объекты на расстоянии от пяти сантиметров до трёх метров. Для промышленного робота, который будет курсировать по цехам производства, этого вполне достаточно. Для робота-уборщика этого вполне достаточно. Данное устройство обойдётся примерно в десять раз ниже по цене, чем покупка лидара. Однако решение на стереокамере будет полезно только в том случае, если допустимо пренебречь качественным изображением карты глубины, поэтому выбор об использовании той или иной технологии нужно делать с умом. 




\section{Текущее состояние проекта и планы по дальнейшему развитию}

На данном этапе разработка основного функционала завершена и устройство можно использовать для решения практических задач. Тем не менее есть некоторые улучшения, которые хотелось бы добавить позже: жёсткая фиксация физических параметров камеры (например, фокусное расстояние), получение конкретного расстояния до объекта, а также автоматический подбор параметров для любой стереокамеры.

Помимо программных компонентов, в условиях жёсткого санкционного давления, хотелось бы портировать данное решение на российскую электронную базу. Таким решением вполне может стать МВ77.07 от компании НТЦ \enquote{Модуль}. Данная плата является микрокомпьютером, так как имеет свой центральный процессор, который в свою очередь поддерживает векторные операции и, предположительно, должен достаточно хорошо работать с матрицами.

\image{mb7707.jpg}{Микрокомпьютер МВ77.07}{0.5}

Однако портирование моего программного продукта под данный процессор подразумевает переписывание всех библиотек на C/C++, что выходит далеко за рамки курсовой работы, поэтому оставляю эту тему как задел на будущее.
