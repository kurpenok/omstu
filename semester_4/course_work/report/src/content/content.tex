\section{Постановка задачи}

Все лучшие решения человеческих проблем подсмотрены у природы. Задача с определением расстояния - не исключение. Так уж вышло, что компьютерное стереозрение является полным аналогом зрения человеческого. Как же оно работает?

Предположим, миг для человека - это конкретная картинка перед глазами с конкретными значениями "пикселов" в матрице координат. Наш мозг буквально "накладывает" данные изображения друг на друга и получает разницу. Чем больше разница, тем объёмнее кажется один и тот же объект.

Следовательно, задача создания прибора сводится к получению разницы между двумя картинками с двух различных камер. Какие подзадачи отсюда возникают? Камеры должны быть жёстко закреплены друг с другом (хотя эта проблема нивелируется аксиомой о двух точках на прямой). Фокусное расстояние у двух камер должно быть одинаковым, чтобы не получать различные значения пикселов (из-за размытия) для одинаковой зоны изображения. Так же требуется провести калибровку камер, так как линзы не являются идеальными и в матрице ошибок требуется указать эти проблемы. Помимо решения проблемы неидеальных линз, калибровка позволяет увидеть какую зону камеры покрывают одновременно - это поможет определиться с разрешением выходной карты глубины.

\section{Решение}

\subsection{Описание математического аппарата стереозрения}

Чертежи, схемы, формулы для определения расстояний.

\subsection{Описание устройства}

Здесь неплохо было бы описать стереокамеру. Физические выводы, разрешение, привести скриншоты технической документации, фотографии самой камеры.

\subsection{Создание API для работы с камерой}

Пример изображения склейки с двух камер.

\subsection{Калибровка камеры}

Условное описание процесса калибровки и зачем оно нужно. Почему шахматная доска? Калибровочные изображения. Итоговая матрица.

\subsection{Получение разницы изображений}

Описание OpenCV фукнции для получения разницы.

\subsection{Настройка параметров получения разницы}

Экран с настройкой.

\subsection{Фиксация параметров. Запуск приложения}

\section{Текущее состояние проекта и планы по дальнейшему развитию}

На данном этапе основной функционал завершён и приложение можно использовать. Однако есть некоторые улучшения, которые хотелось бы добавить: жёсткая фиксация параметров камеры, получение конкретного расстояния до объекта, автоматической подбор параметров для любой стереокамеры. 
