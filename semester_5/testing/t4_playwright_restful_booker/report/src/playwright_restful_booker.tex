\documentclass[a4paper]{article}

\usepackage[utf8]{inputenc}
\usepackage[14pt]{extsizes}
\usepackage[russian]{babel}
\usepackage[left=20mm, top=15mm, right=15mm, bottom=15mm, nohead, footskip=10mm]{geometry}
\usepackage{graphicx}
\usepackage{listings}
\usepackage{minted}
\usepackage{xcolor}

\setlength{\parindent}{0pt}

\begin{document}
    \thispagestyle{empty}

    \begin{center}
        Министерство науки и высшего образования Российской Федерации

        Федеральное государственное автономное образовательное учреждение высшего образования

        <<Омский государственный технический университет>>

        \vspace{1cm}
        Факультет информационных технологий и компьютерных систем

        Кафедра <<Прикладная математика и фундаметральная информатика>>

        \vspace{3cm}
        \textbf{Расчётно-графическая работа}

        по дисциплине <<Практикум по программированию>>
    \end{center}

    \vspace{3cm}
    \begin{flushright}    
        \begin{tabular}{ r r }
            Студента & Курпенова Куата Ибраимовича \\
            \cline{2-2}
            & \tiny{фамилия, имя, отчество полностью} \\

            Курс & 3, группа ФИТ-222 \\
            \cline{2-2}
            Направление & 02.03.02 Фундаментальная информатика\\
            \cline{2-2}
            & и информационные технологии\\
            \cline{2-2}
            & \tiny{код, наименование} \\

            Руководитель & асс. \\
            \cline{2-2}
            & \tiny{должность, ученая степень, звание} \\
            & Цифля А. А. \\
            \cline{2-2}
            & \tiny{фамилия, инициалы} \\

            Выполнил & \\
            \cline{2-2}
            & \tiny{дата, подпись студента} \\

            Проверил & \\
            \cline{2-2}
            & \tiny{дата, подпись руководителя} \\

        \end{tabular}
    \end{flushright}

    \vspace*{\fill}
    \begin{center}
        Омск 2024
    \end{center}

    \newpage

    \tableofcontents

    \newpage

    \section{Основная часть}

    \subsection{Подготовка окружения}

    Для создания виртуального окружения используем модуль \texttt{virtualenv}.
    Создание и активация окружения выполняется следующими командами:

    \begin{minted}[frame=single]{bash}
virtualenv3 venv && source ./venv/bin/activate    
    \end{minted}

    После активации виртуального окружения требуется установить модули для тестирования
    и для работы с \texttt{JSON}-схемами:

    \begin{minted}[frame=single]{bash}
pip3 install pytest-playwright jsonschema
    \end{minted}

    Чтобы была возможность повторно развернуть идентичное виртуальное окружение,
    зафиксируем требуемые зависимости в файле \texttt{requirements.txt}:

    \begin{minted}[frame=single]{bash}
pip3 freeze > requirements.txt
    \end{minted}

    Для запуска тестов командой \texttt{pytest} создадим пустой файл \texttt{conftest.py},
    чтобы определить корневую директорию проекта:

    \begin{minted}[frame=single]{bash}
touch conftest.py
    \end{minted}

    Виртуальное окружение готово к использованию!

    \subsection{Создание \texttt{API}-клиента}

    Так как одним из условий задания было конструирование запросов вне тест-кейсов,
    создадим файл с классом, содержащим функции для обращения к \texttt{API} сервиса:

    \begin{minted}[frame=single]{python}
# Библиотека для работы с запросами
import requests


class RestfulBookerAPI:
    BASE_URL = "https://restful-booker.herokuapp.com"

    # Функция для получения токена через API
    def get_token(self, payload):
        url = f"{self.BASE_URL}/auth"
        response = requests.post(url, json=payload)
        return response
    # Функция для создания бронирования через API
    def create_booking(self, payload):
        url = f"{self.BASE_URL}/booking"
        response = requests.post(url, json=payload)
        return response

    # Функция для удаления бронирования
    def delete_booking(self, booking_number):
        url = f"{self.BASE_URL}/booking/{booking_number}"
        response = requests.delete(url)
        return response
    \end{minted}

    \subsection{Тестирование \texttt{API}-хэндлера получения токена}

    Для тестирования процесса получения токена нам нужно создать объект класса
    \texttt{RestfulBookingAPI}, вызвать метод для создания токена и сверить
    \texttt{JSON}-схему ответа сервиса с \texttt{JSON}-схемой ответа, указанной в документации:

    \begin{minted}[frame=single]{python}
from jsonschema import validate

from api_client import RestfulBookerAPI


class TestAuthToken:
    def setup_class(self):
        self.api = RestfulBookerAPI()

    def test_get_token_success(self):
        payload = {"username": "admin", "password": "password123"}

        schema = {
            "type": "object",
            "properties": {"token": {"type": "string"}},
            "required": ["token"],
        }

        response = self.api.get_token(payload)

        assert (
            response.status_code == 200
        ), f"Expected status code 200, got {response.status_code}"

        response_json = response.json()
        validate(instance=response_json, schema=schema)
    \end{minted}

    \subsection{Тестирование \texttt{API}-хэндлера бронирования}

    Тестирование создания бронирования несколько отличается от тестирования получения токена:
    у запроса другая схема ответа сервиса бронирования, а также требуется удаление бронирования
    после завершения тестирования, так как тест не должен оставлять после себя сущности на сервере:

    \begin{minted}[frame=single]{python}
from jsonschema import validate

from api_client import RestfulBookerAPI


class TestCreateBooking:
    def setup_class(self):
        self.api = RestfulBookerAPI()

    def test_create_booking_success(self):
        payload = {
            "firstname": "John",
            "lastname": "Doe",
            "totalprice": 123,
            "depositpaid": True,
            "bookingdates": {
                "checkin": "2024-01-01",
                "checkout": "2024-01-10",
            },
            "additionalneeds": "Breakfast",
        }

        schema = {
            "type": "object",
            "properties": {
                "bookingid": {"type": "integer"},
                "booking": {
                    "type": "object",
                    "properties": {
                        "firstname": {"type": "string"},
                        "lastname": {"type": "string"},
                        "totalprice": {"type": "integer"},
                        "depositpaid": {"type": "boolean"},
                        "bookingdates": {
                            "type": "object",
                            "properties": {
                                "checkin": {"type": "string"},
                                "checkout": {"type": "string"},
                            },
                            "required": ["checkin", "checkout"],
                        },
                        "additionalneeds": {"type": "string"},
                    },
                    "required": [
                        "firstname",
                        "lastname",
                        "totalprice",
                        "depositpaid",
                        "bookingdates",
                        "additionalneeds",
                    ],
                },
            },
            "required": ["bookingid", "booking"],
        }

        response = self.api.create_booking(payload)

        assert (
            response.status_code == 200
        ), f"Expected status code 200, got {response.status_code}"

        response_json = response.json()
        validate(instance=response_json, schema=schema)

        self.api.delete_booking(response.json()["bookingid"])
    \end{minted}

    \subsection{Запуск тестирования и проверка результатов}

    После запуска тестов командой \texttt{pytest}, получаем следущий вывод:

    \begin{minted}[frame=single]{bash}
======================= test session starts =======================
platform linux -- Python 3.12.7, pytest-8.3.3, pluggy-1.5.0
rootdir: .../omstu/semester_5/testing/t4_playwright_restful_booker
plugins: base-url-2.1.0, playwright-0.5.2
collected 2 items                                        

tests/test_auth.py .                                         [ 50%]
tests/test_booking.py .                                      [100%]

======================== 2 passed in 3.80s ========================
    \end{minted}

\end{document}
