\documentclass[a4paper]{article}
\usepackage[14pt]{extsizes}
\usepackage[utf8]{inputenc}
\usepackage[russian]{babel}
\usepackage[left=20mm, top=15mm, right=15mm, bottom=15mm, nohead, footskip=10mm]{geometry}

\begin{document}
    \thispagestyle{empty}

    \begin{center}
        Министерство науки и высшего образования Российской Федерации

        Федеральное государственно автономное образовательное учреждение высшего образования

        <<Омский государственный технический университет>>

        \vspace{1cm}
        Факультет информационных технологий и компьютерных систем

        Кафедра <<Прикладная математика и фундаметральная информатика>>

        \vspace{3cm}
        \textbf{Реферат}

        по дисциплине <<Физическая культура>>
    \end{center}
    
    \vspace{3cm}
    \begin{flushright}    
        \begin{tabular}{ r r }
            Студента & Курпенова Куата Ибраимовича \\
            \cline{2-2}
            & \tiny{фамилия, имя, отчество полностью} \\

            Курс & 3, группа ФИТ-222 \\
            \cline{2-2}
            Направление & 02.03.02 Прикладная математика \\
            \cline{2-2}
            & и фундаментальная информатика \\
            \cline{2-2}
            & \tiny{код, наименование} \\
            
            Руководитель & доц., канд. пед. наук \\
            \cline{2-2}
            & \tiny{должность, ученая степень, звание} \\
            & Павлютина Л. Ю. \\
            \cline{2-2}
            & \tiny{фамилия, инициалы} \\
            
            Выполнил & \\
            \cline{2-2}
            & \tiny{дата, подпись студента} \\
            
        \end{tabular}
    \end{flushright}
    
    \vspace*{\fill}
    \begin{center}
        Омск 2024
    \end{center}

    \newpage

    \tableofcontents

    \newpage

    \section{Физическая подготовка в Вооруженных Силах}

    \subsection{Введение}

    Профессиональная физическая культура в Вооруженных силах Российской Федерации представляет собой одно из важных направлений системы физического воспитания, призванное формировать определенные прикладные знания, качества, умения и навыки, способствующие достижению объективной готовности человека к успешной профессиональной деятельности. Из этого следует, что уровень подготовленности является важнейшим интегральным показателем, напрямую влияющим на эффективность его становления как специалиста. Поэтому дальнейшим нашим исследованием стало выявление эффективности применения средств и методов профессиональной физической культуры для развития профессиональных качеств в рядах Вооруженных сил и для дальнейшей перспективы применения профессионального навыка. Профессиональное обучение требует проявления специфических качеств, что сопряжено с формированием многочисленных и разнородных по составу приспособительных механизмов. Отдельные профессиональные качества имеют преимущественное включение во все виды учебной деятельности. Степень их развития и определяет уровень готовности студентов к службе в Вооруженных силах РФ.

    В связи с поставленной целью нами были сформированы основные задачи подготовленности студентов, в основу которых были положены следующие общеметодические и специальные принципы профессиональной физической культуры:

    \begin{itemize}
        \item формирование специальных знаний, профессиональных умений и навыков;
        \item развитие профессионально значимых физических качеств, психомоторных способностей и волевых качеств;
        \item воспитание осознанного отношения к занятиям физическими упражнениями;
        \item приобретение личного опыта повышения двигательных и функциональных возможностей;
        \item обеспечение общей и профессионально-прикладной физической подготовленности к будущей профессии;
        \item совершенствование уровня психофизической работоспособности на протяжении всего периода обучения.
    \end{itemize}

    \subsubsection{Значение, цель, задачи физической культуры в Вооруженных Силах}

    Физическая культура является одним из основных предметов боевой подготовки, важной и неотъемлемой частью воинского обучения и воспитания военнослужащих Вооруженных Сил. Физическая подготовка в Вооруженных Силах является неотъемлемой частью системы воспитания в нашем государстве и строится на базе и в соответствии с его общими принципами. Но при этом ее содержание направлено на формирование умений и навыков, развитие физических качеств военнослужащих, необходимых прежде всего для эффективного выполнения поставленных задач по их боевому предназначению. На основании выше сказанного приходим к выводу, что задачи, решаемые физической подготовкой военнослужащих имеют прямую связь с безопасностью страны. Кроме того способствуют всестороннему развитию личности и патриотическому воспитанию молодежи.

    Физическая подготовка и спорт в Вооруженных Силах является одним из основных предметов боевой и профессионально-должностной подготовки, важной и неотъемлемой частью военного обучения и воспитания военнослужащих.

    Цель физической подготовки заключается в обеспечении необходимого уровня физической подготовленности военнослужащих для эффективного выполнения поставленных задач по их боевому предназначение в любое время и в любых условиях.

    Общими задачами физической подготовки являются:

    \begin{itemize}
        \item развитие и поддержание на надлежащем уровне выносливости, силы, быстроты и ловкости;
        \item овладение навыками в передвижении по пересеченной местности в пешем порядке и на лыжах, преодолении естественных препятствий, рукопашного боя, военно-прикладного плавания;
        \item воспитание психической устойчивости, уверенности в своих силах, целеустремленности и решительности, инициативы и находчивости, настойчивости и упорства, выдержанности и самообладания;
        \item крепление здоровья, закаливание и повышение устойчивости организма к воздействию неблагоприятных факторов боевой деятельности;
        \item формирование здорового образа жизни и гармоничного физического развития.
    \end{itemize}

    Специальные задачи физической подготовки определяются в зависимости от специфики военно-профессиональной деятельности военнослужащих.

    Физическая подготовка военнослужащих направлена на:

    \begin{itemize}
        \item проходящих военную службу по контракту, поддержание их физической подготовленности в соответствие с требованиями военной службы, ее организации и проведения;
        \item срочной военной службы и проходящих службу в резерве Вооруженных Сил - решение общих и специальных задач, связанных с особенностями их боевого предназначения, повышение и поддержание высокого уровня работоспособности, привитие потребности к систематическим занятиям физическими упражнениями и спортом;
        \item курсантов, слушателей военных учебных заведений - решение общих и специальных задач, овладение теоретическими знаниями, организаторско - методическими умениями при проведении занятий всех форм физической подготовки;
        \item физическая подготовка военнослужащих-женщин, проходящих военную службу по контракту - повышение общей физической подготовленности, укрепление здоровья с учетом характера служебной деятельности и особенности организма.
    \end{itemize}

    \subsection{Специальная направленность физической подготовки в различных видах Вооруженных Сил}

    Физическая подготовка в воинской части организуется и проводится с учетом особенностей ее предназначения и имеет специальную направленность.

    Целью специальной направленности физической подготовки в воинских частях видов Вооруженных Сил является - развитие у военнослужащих наиболее важных для военной специальности физических и специальных качеств, военно-прикладных навыков на базе их физической подготовленности.

    Специальными задачами физической подготовки в Сухопутных войсках являются:

    \begin{itemize}
        \item для военнослужащих мотострелковых подразделений и подразделений курсантов военных учебных заведений, готовящих офицеров для мотострелковых подразделений;
        \item для военнослужащих танковых, артиллерийских, автомобильных, инженерных, транспортных подразделений и подразделений курсантов военных учебных заведений, готовящих офицеров для этих подразделений;
        \item для военнослужащих подразделений связи, радиотехнических, ракетных, зенитных ракетных подразделений и подразделений курсантов военных учебных заведений, готовящих офицеров для этих подразделений;
        \item для военнослужащих подразделений обеспечения, обслуживания, ремонта, баз хранения вооружения и техники, подразделений курсантов военных учебных заведений, готовящих офицеров для этих подразделений;
        \item для летного состава и курсантов военных учебных заведений военно-воздушных сил и войск противовоздушной обороны;
        \item для военнослужащих подразделений аэродромно-технического обеспечения:
        \item для военнослужащих разведывательных подразделений, подразделений сил специальных операций Вооруженных Сил.
    \end{itemize}

    \subsection{Формы физической подготовки военнослужащих}
    
    Физическая подготовка организуется и проводится в следующих формах:

    \begin{itemize}
        \item учебные занятия;
        \item утренняя физическая зарядка;
        \item спортивно-массовая работа;
        \item физическая тренировка в процессе учебно-боевой деятельности;
        \item самостоятельная физическая тренировка.
    \end{itemize}

    \subsubsection{Учебные занятия}

    Учебные занятия по физической подготовке занимают важное место в общей системе боевой подготовки военнослужащих. Официально это выражается в признании за физической подготовкой статуса одного из основных предметов боевой подготовки войск.

    Учебные занятия по физической подготовке ориентированы на развитие и постоянное совершенствование физических качеств, формирование и совершенствование военно-прикладных двигательных навыков, улучшение физического развития, укрепление здоровья и повышение устойчивости организма военнослужащих к воздействию неблагоприятных факторов военно-профессиональной деятельности.

    Особенности учебных занятий:

    \begin{itemize}
        \item приоритетность в решении задач физического совершенствования военнослужащих. Учебные занятия по физической подготовке располагают наибольшими возможностями по воздействию на личный состав. Им присущи практически все функции, которыми обладает физическая подготовка в целом как составная часть системы боевого совершенствования войск;
        \item обязательность учебных занятий для всех военнослужащих. Все военнослужащие независимо от служебного положения, звания и состояния здоровья привлекаются к регулярным занятиям по физической подготовке.
    \end{itemize}

    На учебных занятиях в более полном объеме отражаются все этапы процесса обучения: ознакомление, разучивание, закрепление, совершенствование.

    Это обеспечивает эффективное овладение новыми физическими упражнениями, формирование прочных военно-прикладных, двигательных навыков, привитие необходимых теоретических знаний и организаторско-теоретических умений, развитие общих физических и специальных качеств.

    Учебные занятия по физической подготовке подразделяются на теоретические и практические.

    Теоретические занятия по физической подготовке - ориентированы на овладение военнослужащими необходимыми знаниями, предусмотренные программой, и проводятся в виде лекций и семинаров.

    Практические занятия - основной вид учебных занятий по физической подготовке. Они должны иметь учебно-тренировочную и методическую направленность, соответственно делятся на учебно-тренировочные и методические.

    Учебно-тренировочные занятия могут быть предметными и комплексными;

    Предметные учебно-тренировочные занятия проводятся по разделам физической подготовки: преодолению препятствий и метанию гранат, рукопашному бою, ускоренному передвижению, лыжной подготовке, атлетической подготовке.

    Комплексные учебно-тренировочные занятия ориентированы на повышение общей физической подготовленности военнослужащих. В их содержание включаются упражнения из нескольких разделов физической подготовки.

    Практические занятия проводятся в составе учебной группы и состоят из трех частей: подготовительной, основной, заключительной.

    Методические занятия по физической подготовке подразделяются на учебно-методические, инструкторско-методические и показные.

    Учебно-методические занятия проводятся с курсантами учебных частей, готовящих командиров отделений, и обеспечивают формирование у них умений, которые необходимы для качественной организации личного состава.

    Инструкторско-методические занятия проводятся в целях повышения знаний, совершенствования умений командиров подразделений в организации и проведению занятий по физической подготовке личного состава. В ходе инструкторско-методических занятий уточняются задачи и содержание предстоящего этапа физического совершенствования личного состава, проверяются знания требований нормативных документов, а кроме того уровень практической подготовленности инструктируемых лиц, проводится разбор организации и методики проведения отдельных частей занятия и методики обучения упражнениям, приемам и действиям, осуществляется демонстрация наиболее сложных моментов организации и проведения физической подготовки с последующей практикой командиров подразделений и разбором их действий, дается оценка теоретической, практической и методической подготовленности командиров подразделений к проведению предстоящих занятий.

    Показные занятия проводятся в целях показа образцовой организации и методики проведения занятий, выработки у командиров подразделений и других должностных лиц единого подхода к применению эффективных средств и методов физической подготовки. Таким образом, учебные занятия, благодаря рассмотренным особенностям, многообразию видов, разнохарактерности их направленности, богатству содержания, вносят решающий вклад в обеспечение физической готовности военнослужащих к боевой деятельности и поэтому по праву считаются основной формой физического совершенствования личного состава армии.

    \subsubsection{Утренняя физическая зарядка}

    Утренняя физическая зарядка проводится в целях систематической физической тренировки военнослужащих. Она направлена на быстрое приведение организма после сна в активное состояние, повышение разносторонней физической подготоленвности, воспитание привычки к ежедневному выполнению физических упражнений, укреплению здоровья и закаливанию организма. Утренняя физическая зарядка - обязательный элемент распорядка дня. Приводится она через 10 минут после подъема военнослужащих (продолжительность от 30 до 50 минут).

    Утренняя физическая зарядка проводится в виде комплексной тренировки с использованием ранее изученных физических упражнений.

    В содержание зарядки включаются:

    \begin{itemize}
        \item общеразвивающие упражнения в движении и на месте;
        \item комплексы вольных упражнений;
        \item руководство боя без оружия;
        \item упражнения в парах;
        \item бег.
    \end{itemize}

    При погодных условиях, не позволяющих проводить зарядку на открытом воздухе, по решению дежурного по воинской части (военному учебному заведению), зарядка может проводиться в казарменных помещениях или в спортивном зале.

    В зимних условиях при низкой температуре зарядка проводится в быстром темпе с выполнением общеразвивающих упражнений в ходьбе и беге.

    Военнослужащие срочной военной службы и курсанты военных учебных заведений занимаются зарядкой в составе подразделений под руководством командиров подразделений, старшин или методически подготовленных сержантов.

    Членам сборных команд воинской части (военного учебного заведения) в период проведения учебно-тренировочных сборов по подготовке к соревнованиям разрешается заниматься зарядкой самостоятельно в спортивной форме одежды по индивидуальному плану.

    \subsubsection{Спортивно-массовая работа}

    Спортивно-массовая работа направлена на повышение уровня физической подготовленности и спортивного мастерства военнослужащих, организацию содержательного досуга личного состава. Это эффективное средство воспитания у военнослужащих воли и стойкости в действиях при максимальных физических нагрузках и психических напряжениях.

    Учебно-тренировочные занятия и сборы по видам спорта проводятся:

    В сборных командах военных частей, военно-учебных заведений. 3-4 раза в неделю продолжительностью до двух часов.

    Спорт высших достижений представляет собой деятельность спортивного комитета Вооруженных Сил, регламентированный законодательством и направленную на подготовку спортсменов высокого класса с целью достижения ими высоких спортивных результатов на соревнованиях и пропаганду спорта.

    Спортивные и военно-спортивные соревнования в воинских частях проводятся согласно правилам с целью вовлечения военнослужащих в занятия военно-прикладными и другими видами спорта, достижение ими высоких спортивных результатов.

    Смотры спортивно - массовой работы проводятся со всеми категориями военнослужащих с целью достижения ими высоких спортивных результатов по упражнениям Военно-спортивного комплекса. По результатам смотра определяется место каждому подразделению и воинской части.

    Спортивные праздники проводятся для привлечения максимального количества военнослужащих и членов их семей к регулярным занятиям спортом и пропаганды здорового образа жизни. Они обычно приурочиваются к официально установленным праздникам, годовщине части и другим знаменательным для военнослужащих датам.

    \subsubsection{Физическая тренировка в процессе учебно-боевой деятельности}

    Физическая тренировка в процессе учебно-боевой деятельности включает в себя выполнение физических упражнений в условиях несения боевого дежурства, попутную физическую тренировку, физическое упражнение при передвижении войск на транспортных средствах, физическую тренировку в полевых условиях.

    Физические упражнения в условиях несения боевого дежурства организуются и проводятся исходя из особенностей военной специальности военнослужащих с целью сохранения ими высокой умственной и физической работоспособности, восполнения дефицита двигательной активности и поддержания физической и психической готовности к экстренным действиям.

    Применяются в виде специально разработанных комплексов, включающих в себя упражнения для отдельных групп мышц, на внимание и координацию, для предупреждения или снятия неблагоприятных воздействий на организм монотонной работы, однообразной позы, ограничения двигательной активности, некоторых факторов внешней среды.

    Подбор физических упражнений осуществляется с учетом особенностей соответствующей военно-профессиональной деятельности военнослужащих (общего режима дежурства, характера работы, количества в ней пауз или перерывов, рабочей позы, внешних условий).

    В содержание комплекса могут включаться:

    \begin{itemize}
        \item Упражнения в потягивании в поясничной части в сочетании с глубоким дыханием;
        \item Бег на месте с поворотами;
        \item Наклоны;
        \item Повороты и вращения головой и туловищем;
        \item Приседания;
        \item Подскоки на месте с различными движениями рук и ног;
        \item Подводя итоги - бег и ходьба на месте.
    \end{itemize}

    Попутная физическая тренировка проводиться с целью повышения физической подготовленности, совершенствования военно-прикладных двигательных навыков и обеспечивает:

    \begin{itemize}
        \item совершенствование способности военнослужащих к быстрому и эффективному передвижению по разнообразной местности в пешем строю или на лыжах в сочетании с преодолением препятствий;
        \item зазвитие общей выносливости и других физических качеств;
        \item повышение военно-специальной подготовленности;
        \item совершенствование навыков использования оружия в условиях значительных физических нагрузок;
        \item подготовку военнослужащих к действиям в защитной одежде и в условиях ограниченной видимости;
        \item формирование навыков коллективных действий;
        \item воспитание военнослужащих в духе коллективизма и взаимной помощи.
    \end{itemize}

    Физические упражнения при передвижении воинских частей на транспортных средствах выполняются с целью поддержания постоянной готовности к действиям, предупреждения утомления, а в зимнее время переохлаждение организма военнослужащих.

    В период выхода воинской части на полигон (лагерь) оборудуются места для занятий физической тренировкой в полевых условиях атлетической подготовкой (брусья, перекладины, спортивный инвентарь, изготовленный из подручных материалов), полосы препятствий, площадка для выполнения приемов рукопашного боя.

    \subsubsection{Самостоятельная физическая тренировка}

    Является одним из резервов повышения эффективности военно-профессиональной деятельности военнослужащих.

    Оптимально организованная регулярная физическая тренировка является средством повышения устойчивости организма к воздействию неблагоприятных факторов, укрепляет нервную систему, способствует согласованности двигательных и вегетативных функций.

    Содержание самостоятельной физической тренировки составляют: оздоровительные бег и ходьба, лыжные прогулки, плавание, упражнения на гимнастических снарядах, тренажерах, упражнения с тяжестями, спортивные игры и единоборства.

    Величина физических нагрузок обязательно должна быть согласована с врачом. Нагрузка считается выбранной правильно, если после занятия ощущается прилив бодрости и энергии.

    Каждый занимающийся должен до и после занятия контролировать свое самочувствие, пульс, степень усталости и другие показатели. Регулирование физической нагрузки в течение недели должно соответствовать поставленным задачам. Если тренировка носит развивающий характер, длительность и интенсивность нагрузки возрастают. При удержании достигнутого уровня тренированности нагрузка должна быть относительно стабильной.

    При развивающей тренировке рекомендуется проводить не менее трех занятий в неделю с интервалами между ними 2-3 дня. Занятия целесообразно проводить в одно и то же время.

    Восстановительная тренировка (физкультурная пауза) должна проводиться несколько раз в день (в рабочие дни).

    При самостоятельных занятиях физическими упражнениями очень важным для оценки оздоровительного эффекта является систематический контроль.

    Контроль в процессе тренировки предоставляет возможность определить текущий уровень физической подготовленности для планирования оптимальных нагрузок, выявить «отстающие» двигательные качества и оценить величину их прироста.

    Самоконтроль сводится к определению величины физической нагрузки на базе ответных реакций организма.

    Информативными показателями интенсивности нагрузки являются субъективные ощущения.

    Физическая нагрузка считается достаточной, если самочувствие удовлетворительное (не ухудшается), сон и аппетит нормальные, пульс через 10 минут после тренировки менее 90 ударов в минуту.

    Физическая нагрузка считается чрезмерной, если самочувствие неудовлетворительное (ухудшается, появляется слабость, боль в области сердца, головная боль), сон и аппетит ухудшаются (отсутствуют), пульс через 10 минут после тренировки более 90 ударов в минуту.

    \subsection{Заключение}

    Физическая культура и спорт являются не только эффективным средством физического развития человека, укрепления и охраны его здоровья, сферой общения и проявления социальной активности людей, разумной формой организации и проведения их досуга, но бесспорно влияют и на другие стороны человеческой жизни.

    Физическая культура и спорт рассматриваются как одно из важнейших средств воспитания человека, гармонически сочетающего в себе духовное богатство, моральную чистоту, физическое совершенство, способствуя формированию разносторонней личности.

    В процессе занятий физической культурой и спортом закаляется воля, характер, совершенствуется умение управлять собой, быстро и правильно ориентироваться в разнообразных сложных ситуациях, своевременно принимать решения, разумно рисковать или воздерживаться от риска, вырабатывается смелость, сила, быстрота, осмотрительность и умение не сдаваться.

    \newpage
    
    \section{Список источников}

    \begin{itemize}
        \item Инструкция о порядке организации физической подготовки и спорта в Вооруженных Силах и транспортных войсках. - Мн. : МО РБ, 2001
        \item Наставление по физической подготовке в Вооруженных Силах. - Мн. : МО РБ, 1993
        \item Положение по физической подготовке и спорту военнослужащих Вооруженных Сил. - Мн. : МО РБ, 2001
        \item Теория и методика физического воспитания /Под редакцией Б.А. Ашмарина. - М.: Просвещение, 1990.
    \end{itemize}
\end{document}
