\documentclass[a4paper]{article}
\usepackage[14pt]{extsizes}
\usepackage[utf8]{inputenc}
\usepackage[russian]{babel}
\usepackage[left=20mm, top=15mm, right=15mm, bottom=15mm, nohead, footskip=10mm]{geometry}

\begin{document}
    \thispagestyle{empty}

    \begin{center}
        Министерство науки и высшего образования Российской Федерации

        Федеральное государственно автономное образовательное учреждение высшего образования

        <<Омский государственный технический университет>>

        \vspace{1cm}
        Факультет информационных технологий и компьютерных систем

        Кафедра <<Прикладная математика и фундаметральная информатика>>

        \vspace{3cm}
        \textbf{Реферат}

        по дисциплине <<Физическая культура>>
    \end{center}
    
    \vspace{3cm}
    \begin{flushright}    
        \begin{tabular}{ r r }
            Студента & Курпенова Куата Ибраимовича \\
            \cline{2-2}
            & \tiny{фамилия, имя, отчество полностью} \\

            Курс & 3, группа ФИТ-222 \\
            \cline{2-2}
            Направление & 02.03.02 Прикладная математика \\
            \cline{2-2}
            & и фундаментальная информатика \\
            \cline{2-2}
            & \tiny{код, наименование} \\
            
            Руководитель & доц., канд. пед. наук \\
            \cline{2-2}
            & \tiny{должность, ученая степень, звание} \\
            & Павлютина Л. Ю. \\
            \cline{2-2}
            & \tiny{фамилия, инициалы} \\
            
            Выполнил & \\
            \cline{2-2}
            & \tiny{дата, подпись студента} \\
            
        \end{tabular}
    \end{flushright}
    
    \vspace*{\fill}
    \begin{center}
        Омск 2024
    \end{center}

    \newpage

    \tableofcontents

    \newpage

    \section{Лечебная физкультура при заболеваниях органов пищеварения}

    \subsection{Введение}
    
    В последние годы заболевания желудочно-кишечного тракта (далее ЖКТ) стали очень распространенными в мире, по сравнению с ситуацией, которая была раньше. Особенно частыми стали случаи среди детского и молодежного населения.

    Такие заболевания приводят к частичному разрушению некоторых органов: желудка, печени, кишечника и поджелудочной железы. Желудочно-кишечный тракт - это система органов, целью которой является переработка и извлечение из пищи всех необходимых организму питательных веществ, содействие всасыванию полученных питательных веществ в кровь, а также удаление из организма остатков пищи, которые не были полностью переварены. ЖКТ отвечает за многие важные функции в организме человека.

    Наиболее распространение болезни ЖКТ получили следующие недуги:
    \begin{itemize}
        \item Язвенная болезнь желудка и двенадцатиперстной кишки (болезнь Крона, язвенный колит).
        \item Синдром раздраженного кишечника.
        \item Воспалительные процессы органов ЖКТ (холецистит, гастрит).
        \item Различные заболевания поджелудочной железы (панкреатит). 
        \item Заболевания печени (цирроз, гепатит).
        \item Дисбактериоз.
    \end{itemize}

    \subsection{Причины появления нарушений}

    Существуют внешние и внутренние причины нарушения работы ЖКТ. К внешним относится:
    \begin{itemize}
        \item Несбалансированное питание. Переедание или прием тяжелой пищи, голодание, перекусы на ходу, несбалансированный рацион.
        \item Употребление вредных жидкостей. К ним можно отнести алкоголь, суррогаты, воды с красителями или даже плохо очищенную воду.
        \item Лекарственные средства. Некоторые медикаменты, в состав которых входят салицилаты (аспирин и др.), приводят к постепенному разрушению слизистой желудка, что ведет к серьезным заболеваниям пищевода.
        \item Курение. Курение воздействует на слизистую оболочку и может привести к появлению спазма желудка и двенадцатиперстной кишки. 
        \item Стресс. Стресс истощает организм в целом, так как активизирует все силы организма для защиты от внешних воздействий, а это приводит к падению иммунитета.
        \item Микроорганизмы.  Сейчас принято считать бактерию спиральной формы одной из основных причин заболеваний органов пищеварения. Главной ее функцией является воздействие на слизистую оболочку ЖКТ, приводящее к снижению ее защитных свойств.
    \end{itemize}

    К внутренним причинам можно отнести: генетические, внутриутробные патологии и аутоиммунные заболевания.

    \subsection{Профилактика}

    Врачи рекомендуют стараться максимально снизить количество опасных рисков, которые могут привести к заболеваниям желудка и кишечника. Профилактические меры и здоровый образ жизни помогут значительно снизить вероятность желудочно-кишечных заболеваний.

    Физические упражнения влияют на функции пищеварения через центрально-нервную систему и моторно-висцеральные рефлексы.

    Умеренные физические нагрузки с небольшим числом повторений стимулируют, интенсивные и длительные, вызывающие утомление - угнетают функции желудочно-кишечного тракта.

    Физические упражнения положительно влияют на регенеративные процессы в слизистой оболочке желудка и двенадцатиперстной кишки. При этом улучшается микроциркуляция в тканях слизистой оболочки, в мышцах, увеличивается объем циркулирующей крови за счет ее выхода из депо (печень, селезенка), что уменьшает застойные явления и воспалительные процессы.

    Специальные задачи лечебной физической культуры (далее ЛФК): нормализация нейрогуморальной регуляции функций ЖКТ (секреторной, моторной и др.); улучшение лимфо- и кровообращения в брюшной полости и малом тазу, профилактика застойных явлений и спаечных процессов; укрепление мышц брюшного пресса и тазового дна; развитие полного дыхания с акцентом на диафрагмальном типе.

    Лечебная физкультура применяется в фазе затухания обострения и в фазе ремиссии. В острой фазе заболевания и при осложнениях занятия физиотерапией следует прекратить. Методика физиотерапии предусматривает сочетание общеразвивающих и специальных упражнений: для мышц брюшного пресса, дыхательных упражнений с акцентом на диафрагмальные, упражнения на расслабление.

    \subsection{Лечебная гимнастика}
    
    Примерный комплекс лечебной гимнастики включает в себя ряд упражнений:
    \begin{itemize}
        \item исходное положение – сидя на стуле, руки опущены, ноги шире плеч. Выполняется наклон корпуса вправо, левая ладонь плавно скользит до мышечной впадины, после чего осуществляется возвращение в исходное положение. То же самое движение выполнятся с наклонов влево. Дыхание не задерживается, темп выполнения упражнения средний. Количество повторений – 4-6 раз в каждую сторону.
        \item исходное положение – то же, руки на поясе. Вдох и левая рука отводится в сторону, в тоже время влево поворачивается корпус и голова, на выдохе возвращаемся в исходное положение. Повторить упражнение 3-4 раза в одну, а затем в другую сторону.
        \item исходное положение – то же самое, кисти рук на затылке. На вдохе поворачиваем корпус и голову влево, немного прогибаясь в грудном отделе позвоночника, на выдохе правый локоть должен достать левое ребро. Возвращаемся в исходное положение и повторяем то же, но в другую сторону. Повторяем 2-3 раза в каждую из сторон.
        \item исходное положение – то же, что и в предыдущих случаях, ладони на коленях. Поочередно подтягиваем одно, а затем второе колено к подбородку, помогая себя руками. Повторить 4-6 раз в среднем темпе.
        \item исходное положение – то же, кисти на плечах. Выполняются круговые движения в плечевых суставах, амплитуда максимальная, сначала осуществляются вращения вперед, затем назад. Повторяем 6-8 раз в среднем темпе.
        \item исходное положение – такое же. На вдохе нужно поднять руки и подтянуться, на выдохе развернуть корпус вправо, в эту же сторону свесить прямые руки. То же самое повторить влево. Количество повторений – 2-3, темп средний.
        \item исходное положение – стоя, ноги на ширине плеч, руки опущены. На выдохе поднимаем согнутую в колене левую ногу, при помощи рук подтягиваем колено к груди. На вдохе возвращаемся в исходное положение. То же самое выполняется другой ногой. Каждой ногой нужно выполнить по 2-3 раза в среднем темпе.
        \item исходное положение – то же. Вдох – поднять прямые руки вверх, немного прогнуться в пояснице, правую ногу 6 отставить назад на носок. То же самое повторить с левой ногой. По 2-3 повторения каждой ногой.
        \item исходное положение – то же, руки поставить на пояс. Вдох – наклон корпуса влево, выдох – возвращение в исходное положение. Повторяем 2-3 раза в каждую сторону.
        \item исходное положение – то же. Выдох – наклон вперед, угол между бедрами и корпусом должен составлять 900. Вдох – исходное положение. Нужно повторить 4-6 раз, средний темп.
        \item исходное положение – то же, одна кисть находится на груди, другая на животе. Вдох – надуть живот, выдох – втянуть. Повторить 3-4 раза в медленном темпе.
    \end{itemize}
    
    Наиболее эффективный показывает комплекс занятий, построенный следующим образом: через 1-1, 5 часа после завтрака выполняется лечебная гимнастика под музыку, продолжительность 30-40 минут, после аутотренинга ЛГ отдых 30-40 минут, затем дозированная ходьба или плавание.

    \newpage
    
    \section{Список источников}

    \begin{itemize}
        \item Соловьёва Н. Г., Тихонова В. И. Учебно-методический комплекс по учебной дисциплине «Лечебная физическая культура при заболеваниях и травмах» // 2019. – 253 с.
        \item Ермилова О. Ю. Лечебная физическая культура при заболеваниях органов пищеварения: методические рекомендации для студентов очной формы обучения и преподавателей кафедры физического воспитания и спорта // 2014. – 14с.
    \end{itemize}
\end{document}
