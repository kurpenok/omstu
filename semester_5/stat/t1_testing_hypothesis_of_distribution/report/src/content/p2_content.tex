\section*{Проверка гипотезы о законе распределения генеральной совокупности}

\subsection*{Данные из выборки}

-71, -44, -58, -49, -53, -72, -59, -59, -62, -78, -62, -47, -46, -74, -62, -52, -45, -66,
-48, -59, -44, -56, -58, -34, -55, -57, -51, -59, -58, -53, -67, -38, -90, -68, -36, -53,
-35, -50, -74, -87, -50, -72, -50, -64, -51, -72, -52, -67, -58, -38, -79, -54, -37, -91,
-62, -73, -62, -37, -61, -60, -64, -71, -64, -63, -72, -51, -47, -47, -49, -60, -74, -42,
-45, -52, -67, -50, -64, -65, -58, -21, -67, -58, -36, -78, -63, -59, -59, -77, -77, -54,
-69, -40, -41, -52, -79, -29, -59, -27, -59, -23, -63, -70, -58, -74, -67, -58, -65, -68,
-59, -46, -58, -73, -58, -50, -72, -35, -57, -55, -68, -79, -58, -56, -62, -63, -39, -66,
-70, -58, -60, -64, -54, -68, -69, -50, -30, -67, -71, -44, -84, -44, -71, -68, -86, -70,
-44, -55, -41, -74, -48, -71, -65, -50, -46, -53, -49, -66, -69, -30, -59, -59, -39, -72,
-65, -74, -79, -70, -48, -45, -69, -56, -54, -60.

\subsection*{Вариационный ряд}

-91, -90, -87, -86, -84, -79, -79, -79, -79, -78, -78, -77, -77, -74, -74, -74, -74, -74,
-74, -73, -73, -72, -72, -72, -72, -72, -72, -71, -71, -71, -71, -71, -70, -70, -70, -70,
-69, -69, -69, -69, -68, -68, -68, -68, -68, -67, -67, -67, -67, -67, -67, -66, -66, -66,
-65, -65, -65, -65, -64, -64, -64, -64, -64, -63, -63, -63, -63, -62, -62, -62, -62, -62,
-62, -61, -60, -60, -60, -60, -59, -59, -59, -59, -59, -59, -59, -59, -59, -59, -59, -58,
-58, -58, -58, -58, -58, -58, -58, -58, -58, -58, -58, -57, -57, -56, -56, -56, -55, -55,
-55, -54, -54, -54, -54, -53, -53, -53, -53, -52, -52, -52, -52, -51, -51, -51, -50, -50,
-50, -50, -50, -50, -50, -49, -49, -49, -48, -48, -48, -47, -47, -47, -46, -46, -46, -45,
-45, -45, -44, -44, -44, -44, -44, -42, -41, -41, -40, -39, -39, -38, -38, -37, -37, -36,
-36, -35, -35, -34, -30, -30, -29, -27, -23, -21.

\subsection*{Интервальный статистический ряд}

Количество интервалов в интервальном статистическом ряде рассчитывается по формуле Стёрджеса:
\[n = 1 + 3.322 \cdot \log_{10}{N}\]

В нашем случае формула принимает вид:
\[1 + 3.322 \cdot \log_{10}{172} = 9\]

\subsection*{Полигон и гистограмма относительных частот}
\image{relative_frequencies_polygon_and_histogram.png}{Графики полигона и гистограммы относительных частот}{1.0}

\subsection*{График эмпирической функции распределения}
\image{empirical_distribution.png}{График эмпирической функции распределения}{0.5}

\subsection*{Числовые характеристики выборки}

Формула для нахождения выборочного среднего:
\[\bar{x} = \frac{1}{n} \sum_{i=1}^{n} x_i\]

Формула для нахождения исправленной выборочной дисперсии:
\[s^2 = \frac{1}{n-1} \sum_{i=1}^{n} (x_i - \bar{x})^2\]

Мода и медиана находятся в программном коде без формул.

Формула для нахождения эксцесса:
\[E(X) = \frac{1}{n} \sum_{i=1}^{n} \left( \frac{x_i - \bar{x}}{s} \right)^4 - 3\]

Формула для нахождения асимметрии:
\[\text{As} = \frac{1}{n s^3} \sum_{i=1}^{n} (x_i - \bar{x})^3\]

\begin{spacing}{1.0}
    \lstinputlisting[caption=Значения параметров выборки]{listings/sample_parameters.txt}
\end{spacing}

\subsection*{Гипотеза о распределении генеральной совокупности}

Я предполагаю, что распределение является нормальным, так как гистограмма относительных частот похожа на нормальное распределение, а график эмпирического распределения похож на сигмоиду.

\subsection*{Оценка параметров нормального распределения генеральной совокупности}

Формула оценки среднего:
\[\hat{\mu} = \frac{1}{n} \sum_{i=1}^{n} x_i\]

Формула оценки стандартного отклонения:
\[\hat{\sigma} = \sqrt{\frac{1}{n-1} \sum_{i=1}^{n} (x_i - \hat{\mu})^2}\]

\subsection*{Теоретические аналоги функции}
\image{theoretical_density.png}{График теоретического аналога плотности распределения}{0.5}
\image{theoretical_distribution.png}{График теоретического аналога эмпирической функции}{0.5}

\subsection*{Выполнение правила \enquote{трёх сигма}}

Правило трёх сигма (или правило трёх стандартных отклонений) используется в статистике для описания распределения данных в нормальном распределении. Оно утверждает, что примерно 68\% данных находятся в пределах одного стандартного отклонения от среднего, около 95\% — в пределах двух стандартных отклонений, и примерно 99.7\% — в пределах трёх стандартных отклонений.

В моём случае границы интервалов имеют значения: -98.3140 и -18.2674.

\subsection*{Проверка по критерию Пирсона}

Критерий Пирсона, также известный как $\chi^2$ (хи-квадрат) тест, используется в статистике для проверки гипотез о распределении категориальных данных. Основная цель этого критерия — определить, насколько наблюдаемые частоты (или количество случаев) в различных категориях соответствуют ожидаемым частотам, которые предполагаются в соответствии с некоторой теоретической моделью.

Формула для расчёта статистики хи-квадрат:
\[ \chi^2 = \sum_{i=1}^{k} \frac{(O_i - E_i)^2}{E_i} \]

\begin{spacing}{1.0}
    \lstinputlisting[caption=Результат проверки по критерию Пирсона]{listings/pearson.txt}
\end{spacing}

Так как $P-value$ меньше уровня значимости, делаем вывод, что нулевая гипотеза верна.

\subsection*{Проверка по критерию Колмогорова}

Критерий Колмогорова (или тест Колмогорова-Смирнова) — это непараметрический статистический тест, который используется для проверки гипотез о распределении данных. Он позволяет оценить, насколько хорошо наблюдаемые данные соответствуют предполагаемому теоретическому распределению, или сравнить два эмпирических распределения.

Формула для статистики теста Колмогорова:
\[ D = \max_{x} |F_n(x) - F(x)| \]

\begin{spacing}{1.0}
    \lstinputlisting[caption=Результат проверки по критерию Колмогорова]{listings/kolmogorov.txt}
\end{spacing}

Так как $P-value$ меньше уровня значимости, делаем вывод, что нулевая гипотеза верна.

\subsection*{Построение доверительного интервала}

Формула для построения доверительного интервала:
\[ \bar{x} \pm z_{\alpha/2} \cdot \frac{\sigma}{\sqrt{n}} \]

Формула для построения доверительного интервала при неизвестной дисперсии:
\[ \bar{x} \pm t_{\alpha/2, n-1} \cdot \frac{s}{\sqrt{n}} \]

В моём случае уровенб значимости равен 0.05 и границы интервала равны -60.2845 и -56.2969.
