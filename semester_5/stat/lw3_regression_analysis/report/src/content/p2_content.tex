\section*{Корреляционно-регрессионный анализ статистических данных}

\section*{Вариант 12}

Результаты наблюдения двумерной случайной величины $(X, Y)$:

\begin{center}
    \begin{tabular}{|c|c|c|c|c|c|c|}
        \hline
        $X/Y$ & 1 & 2 & 3 & 4 & 5 & 6 \\
        \hline
        1 & 2 & 1 & - & - & - & - \\
        \hline
        2 & 1 & 2 & - & - & - & - \\
        \hline
        3 & - & 3 & 1 & - & - & - \\
        \hline
        4 & - & 1 & 3 & 1 & - & - \\
        \hline
        5 & - & - & 2 & 2 & 2 & 1 \\
        \hline
        6 & - & - & - & 1 & 1 & 1 \\
        \hline
    \end{tabular}
\end{center}

Задание:

\begin{enumerate}
    \item Найти групповое среднее $\overline{y_i}$ переменной $Y$.
    \item В прямоугольной системе координат построить точки $(x_i, \overline{y_i})$ и ломаную линию регрессии $Y$ на $X$.
    \item Найти генеральные средние $\overline{x}$ и $\overline{y}$.
    \item Составить уравнение линейной регрессии $X$ на $Y$ и $Y$ на $X$. Построить график регрессии.
    \item По выбранному значению переменной $X$ сделать прогноз ожидаемого среднего значения переменной.
    \item Установить тесноту связи между переменными величинами $X$ и $Y$.
    \item Оценить существенность выборочного коэффициента корреляции.
    \item Найти 95\%-й доверительные интервалы для среднего значения и коэффициентов уравнения регрессии.
\end{enumerate}

\newpage

\section{Найти групповое среднее}

\begin{enumerate}
    \item $x_1 = 1; y_1 = \frac{1 \cdot 2 + 2 \cdot 1}{1 + 2} = \frac{4}{3}$
    \item $x_2 = 2; y_2 = \frac{1 \cdot 1 + 2 \cdot 2}{1 + 2} = \frac{5}{3}$
    \item $x_3 = 3; y_3 = \frac{2 \cdot 3 + 3 \cdot 1}{3 + 1} = \frac{9}{4}$
    \item $x_4 = 4; y_4 = \frac{2 \cdot 1 + 3 \cdot 3 + 4 \cdot 1}{1 + 3 + 1} = \frac{15}{5} = 3$
    \item $x_5 = 5; y_5 = \frac{3 \cdot 2 + 4 \cdot 2 + 5 \cdot 2 + 6 \cdot 1}{2 + 2 + 2 + 1} = \frac{30}{7}$
    \item $x_6 = 6; y_6 = \frac{4 \cdot 1 + 5 \cdot 1 + 6 \cdot 1}{1 + 1 + 1} = \frac{15}{3} = 5$
\end{enumerate}

\section{Точки и линия регрессии}

\image{regression_line.jpg}{Ломаная линия регрессии}{0.75}

\section{Найти генеральные средние}

$$\overline{x} = \frac{1}{n} \sum^{n}_{i=1}{x_i \cdot n_{x_i}} = \frac{1}{25} (1 \cdot 3 + 2 \cdot 3 + 3 \cdot 4 + 4 \cdot 5 + 5 \cdot 7 + 6 \cdot 3) = \frac{94}{25} = 3.76$$

$$\overline{y} = \frac{1}{n} \sum^{n}_{j=1}{y_i \cdot n_{y_i}} = \frac{1}{25} (1 \cdot 3 + 2 \cdot 7 + 3 \cdot 6 + 4 \cdot 4 + 5 \cdot 3 + 6 \cdot 2) = \frac{78}{25} = 3.12$$

\section{Составить уравнение линейной регрессии}

$$\overline{x^2} = \frac{1}{n} \sum^{n}_{i=1}{{x_i}^2 \cdot n_{x_i}} = \frac{1}{25} (1^2 \cdot 3 + 2^2 \cdot 3 + 3^2 \cdot 4 + 4^2 \cdot 5 + 5^2 \cdot 7 + 6^2 \cdot 3) = \frac{414}{25} = 16.56$$

$$\overline{y^2} = \frac{1}{n} \sum^{n}_{j=1}{{y_i}^2 \cdot n_{y_i}} = \frac{1}{25} (1^2 \cdot 3 + 2^2 \cdot 7 + 3^2 \cdot 6 + 4^2 \cdot 4 + 5^2 \cdot 3 + 6^2 \cdot 2) = \frac{296}{25} = 11.8$$

\begin{align*}
     \overline{xy} &= \frac{1}{n} \sum^{n}_{j=1} \sum^{n}_{i=1} x_i y_i n_{ij} = \frac{1}{25} (1 \cdot 1 \cdot 2 + 1 \cdot 2 \cdot 1 + 2 \cdot 1 \cdot 1 + 2 \cdot 2 \cdot 2 + 3 \cdot 2 \cdot 3 \nonumber \\
    &+ 3 \cdot 3 \cdot 1 + 4 \cdot 2 \cdot 1 + 4 \cdot 3 \cdot 3 + 4 \cdot 4 \cdot 1 + 5 \cdot 3 \cdot 2 + 5 \cdot 4 \cdot 2 + 5 \cdot 5 \cdot 2 \nonumber \\
    &+ 5 \cdot 6 \cdot 1 + 6 \cdot 4 \cdot 1 + 6 \cdot 5 \cdot 1 + 6 \cdot 6 \cdot 1) = \frac{341}{25} = 13.64
\end{align*}

Уравнение $Y$ на $X$:

$$cov(X, Y) = \overline{xy} - \bar{x} \bar{y} = 13.64 - 3.76 \cdot 3.12 = 1.91$$

$${\sigma_x}^2 = \overline{x^2} - \bar{x}^2 = 16.56 - (3.76)^2 = 2.42$$

$$a = \rho_{yx} = \frac{1.91}{2.42} = 0.79$$

$$b = \overline{y} - a \overline{x} = 3.12 - 0.79 \cdot 3.76 = 0.15$$

$$y = ax + b = 0.79x + 0.15$$

Уравнение $X$ на $Y$:

$$cov(X, Y) = \overline{xy} - \bar{x} \bar{y} = 13.64 - 3.76 \cdot 3.12 = 1.91$$

$${\sigma_y}^2 = \overline{y^2} - \bar{y}^2 = 11.8 - (3.12)^2 = 2.0656$$

$$c = \rho_{xy} = \frac{1.91}{2.0656} = 0.925$$

$$d = \overline{x} - c \overline{y} = 3.76 - 0.925 \cdot 3.12 = 0.873$$

$$x = cy + d = 0.925y + 0.873$$

\image{regression.jpg}{Графики уравнений}{0.75}

\section{Прогноз переменной}

При $x=3.76$, $\overline{y}$ равен: $\overline{y} = 0.79 * 3.76 + 0.15 = 3.1204$

\section{Установить тесноту связи}

$$\sigma_x = \sqrt{{\sigma_x^2}} = 1.556$$

$$\sigma_y = \sqrt{{\sigma_y^2}} = 1.437$$

$$r = \frac{\overline{xy} - \bar{x} \bar{y}}{\sigma_x \sigma_y}= \frac{13.64 - 3.76 \cdot 3.12}{1.556 \cdot 1.437} = 0.854$$

По шкале Чеддока оценка силы связи тесная.

\section{Оценить существенность выборочного коэффициента корреляции}

Примем, что: $H_0: r = 0$ и $H_1: r \neq 0$, тогда:

$$|t| = \frac{|r| \sqrt{n - 2}}{\sqrt{1 - r^2}} = \frac{|0.854| \cdot \sqrt{25 - 2}}{\sqrt{1 - 0.854^2}} = 7.885$$

При $\alpha = 0.05$, $t_{0.05;23} = 2.07$ по таблице Стьюдента.

7.88 > 2.07, следовательно, $H_0$ отвергается и признаётся статистическая значимость и надёжность уравнения. Линейная корреляционная связь между переменными присутствует.

\section{Найти доверительный интервал}

Найдём выборочную остаточную дисперсию:

$$S^2 = \frac{1}{n - 2} \sum^n_{i=1}{(\overline{y_i} - y_i)^2} = \frac{1}{23} (\frac{1}{9} + \frac{1}{9} + \frac{9}{16} + 1 + \frac{25}{49} + 1) = 0.143$$

$$S = \sqrt{S^2} = \sqrt{0.143} = 0.378$$

Размах доверительных интервалов:

\begin{align*}
    &\Delta a = t_{a;n - 2} \frac{S}{\sqrt{\sum^n_{i=1}{(x_i - \overline{x})^2}}} = \nonumber \\
    &2.07 \frac{0.378}{\sqrt{(-2.76)^2 + (-1.76)^2 + (-0.76)^2 + 0.24^2 + 1.24^2 + 2.24^2}} = 0.185
\end{align*}

$$\Delta b = \Delta a \sqrt{\overline{x}} = 0.185 \cdot \sqrt{3.76} = 0.35$$

Получаем: $0.606 < a < 0.974$ и $-0.21 < b < 0.51$.

$${{S_y}^2}_{x=1} = S^2 \left( \frac{1}{n} + \frac{(x_0 - \overline{x})^2}{\sum^n_{i=1}{(x_i - \overline{x})^2}} \right) = 0.378 \left( \frac{1}{25} + \frac{(1 - 3.76)^2}{17.9056} \right) = 0.176 $$

$${S_y}_{x=1} = 0.4194$$

$$t_{a;n} = t_{0.05; 25} = 2.07$$

Доверительный интервал при $x=1$: $0.0718 < \overline{y} < 1.808$.
