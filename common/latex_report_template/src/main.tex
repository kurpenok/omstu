% Аrticle использует только шрифты с размером 10-12pt
% Extarticle позволяет использть шрифты любых размеров
\documentclass[14pt, a4paper]{extarticle}

% Языковой пакет. Альтернатива пакету babel
\usepackage{polyglossia}

% Для выделения жёлтым. Можно удалить
\usepackage{xcolor}

% Поля по ГОСТ 2021
\usepackage[left=3cm, right=1.5cm, top=2cm, bottom=2cm]{geometry}

% Перенос при переполнении
\emergencystretch=25pt

% Локализация документа
\setmainlanguage[spelling=modern]{russian}
\setotherlanguage{english}

% Неизменная классика
\setmainfont{Times New Roman}
\setmonofont{JetBrains Mono}
\newfontfamily \cyrillicfont{Times New Roman}
\newfontfamily \cyrillicfonttt[Scale=0.75]{JetBrains Mono}

% Абзацные отступы
\usepackage{indentfirst}
\setlength{\parindent}{1.25cm}

% Пакет для поддержки математических символов
\usepackage{amsmath}

% Пакеты для листингов кода
\usepackage{minted}
\usepackage{listings}

% Пакеты для использования графики
\usepackage{graphicx}
\usepackage{chngcntr}
\graphicspath{{./images/}}
\counterwithin{figure}{section}

% Пакеты для поддержки URL-ссылок
\usepackage{url}
\usepackage{hyperref}
\urlstyle{same}
\hypersetup{
    colorlinks=true,
    linkcolor=black,
    filecolor=blue,
    citecolor = black,
    urlcolor=blue,
}

% Пакет для поддержки цитирования
\usepackage{natbib}
\bibliographystyle{unsrtnat}
\setcitestyle{authoryear, open={(},close={)}}

% Пакет для поддержки кавычек-ёлочек
\usepackage{csquotes}

\begin{document}

% Титульный лист
\include{content/title}

% Включение междустрочных интервалов 1.5 см
% На титульном листе интервалы 1 см
\linespread{1.5}

% Содержание (toc - table of content)
\include{content/toc}

% Контент
\section{Постановка задачи}

Все лучшие решения человеческих проблем подсмотрены у природы. Задача с определением расстояния - не исключение. Так уж вышло, что компьютерное стереозрение является полным аналогом зрения человеческого. Как же оно работает?

Предположим, миг для человека - это конкретная картинка перед глазами с конкретными значениями "пикселов" в матрице координат. Наш мозг буквально "накладывает" данные изображения друг на друга и получает разницу. Чем больше разница, тем объёмнее кажется один и тот же объект.

Следовательно, задача создания прибора сводится к получению разницы между двумя картинками с двух различных камер. Какие подзадачи отсюда возникают? Камеры должны быть жёстко закреплены друг с другом (хотя эта проблема нивелируется аксиомой о двух точках на прямой). Фокусное расстояние у двух камер должно быть одинаковым, чтобы не получать различные значения пикселов (из-за размытия) для одинаковой зоны изображения. Так же требуется провести калибровку камер, так как линзы не являются идеальными и в матрице ошибок требуется указать эти проблемы. Помимо решения проблемы неидеальных линз, калибровка позволяет увидеть какую зону камеры покрывают одновременно - это поможет определиться с разрешением выходной карты глубины.

\section{Решение}

\subsection{Описание математического аппарата стереозрения}

Чертежи, схемы, формулы для определения расстояний.

\subsection{Описание устройства}

Здесь неплохо было бы описать стереокамеру. Физические выводы, разрешение, привести скриншоты технической документации, фотографии самой камеры.

\subsection{Создание API для работы с камерой}

Пример изображения склейки с двух камер.

\subsection{Калибровка камеры}

Условное описание процесса калибровки и зачем оно нужно. Почему шахматная доска? Калибровочные изображения. Итоговая матрица.

\subsection{Получение разницы изображений}

Описание OpenCV фукнции для получения разницы.

\subsection{Настройка параметров получения разницы}

Экран с настройкой.

\subsection{Фиксация параметров. Запуск приложения}

\section{Текущее состояние проекта и планы по дальнейшему развитию}

На данном этапе основной функционал завершён и приложение можно использовать. Однако есть некоторые улучшения, которые хотелось бы добавить: жёсткая фиксация параметров камеры, получение конкретного расстояния до объекта, автоматической подбор параметров для любой стереокамеры. 


% Список использованных источников
\newpage

% Эта строка заменяет название References
\renewcommand* \refname{Список использованных источников}

\nocite{*}
\bibliography{references}
\addcontentsline{toc}{section}{Список использованных источников}


\end{document}
